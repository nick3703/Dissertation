%%%%%%%%%%%%%%%%%%%%%%%%%%%%%%%%%%%%%%%%%
% Beamer Presentation
% LaTeX Template
% Version 1.0 (10/11/12)
%
% This template has been downloaded from:
% http://www.LaTeXTemplates.com
%
% License:
% CC BY-NC-SA 3.0 (http://creativecommons.org/licenses/by-nc-sa/3.0/)
%
%%%%%%%%%%%%%%%%%%%%%%%%%%%%%%%%%%%%%%%%%

%----------------------------------------------------------------------------------------
%	PACKAGES AND THEMES
%----------------------------------------------------------------------------------------

\documentclass{beamer}

\mode<presentation> {

% The Beamer class comes with a number of default slide themes
% which change the colors and layouts of slides. Below this is a list
% of all the themes, uncomment each in turn to see what they look like.

%\usetheme{default}
%\usetheme{AnnArbor}
%\usetheme{Antibes}
%\usetheme{Bergen}
%\usetheme{Berkeley}
%\usetheme{Berlin}
%\usetheme{Boadilla}
%\usetheme{CambridgeUS}
%\usetheme{Copenhagen}
%\usetheme{Darmstadt}
%\usetheme{Dresden}
%\usetheme{Frankfurt}
%\usetheme{Goettingen}
%\usetheme{Hannover}
%\usetheme{Ilmenau}
%\usetheme{JuanLesPins}
%\usetheme{Luebeck}
\usetheme{Madrid}
%\usetheme{Malmoe}
%\usetheme{Marburg}
%\usetheme{Montpellier}
%\usetheme{PaloAlto}
%\usetheme{Pittsburgh}
%\usetheme{Rochester}
%\usetheme{Singapore}
%\usetheme{Szeged}
%\usetheme{Warsaw}

% As well as themes, the Beamer class has a number of color themes
% for any slide theme. Uncomment each of these in turn to see how it
% changes the colors of your current slide theme.

%\usecolortheme{albatross}
%\usecolortheme{beaver}
%\usecolortheme{beetle}
%\usecolortheme{crane}
%\usecolortheme{dolphin}
%\usecolortheme{dove}
%\usecolortheme{fly}
%\usecolortheme{lily}
%\usecolortheme{orchid}
%\usecolortheme{rose}
%\usecolortheme{seagull}
%\usecolortheme{seahorse}
%\usecolortheme{whale}
%\usecolortheme{wolverine}

%\setbeamertemplate{footline} % To remove the footer line in all slides uncomment this line
%\setbeamertemplate{footline}[page number] % To replace the footer line in all slides with a simple slide count uncomment this line

%\setbeamertemplate{navigation symbols}{} % To remove the navigation symbols from the bottom of all slides uncomment this line
}

\usepackage{graphicx} % Allows including images
\usepackage{booktabs} % Allows the use of \toprule, \midrule and \bottomrule in tables
\usepackage{amsmath}
\usepackage{verbatim}
\usepackage{tikz}
\usepackage{amsmath,amsfonts} % Math packages
\usepackage{listings}
\usepackage{appendixnumberbeamer}
\usepackage{xcolor}
\usepackage{amsthm}

\makeatletter
\newenvironment<>{proofs}[1][\proofname]{%
	\par
	\def\insertproofname{#1\@addpunct{.}}%
	\usebeamertemplate{proof begin}#2}
{\usebeamertemplate{proof end}}
\makeatother



\newcommand{\highlight}[1]{%
	\colorbox{orange!50}{$\displaystyle#1$}}


\newcommand{\highlighter}[1]{%
	\colorbox{blue!30}{$\displaystyle#1$}}

\usetikzlibrary{arrows,chains,matrix,positioning,scopes,fit}
%
\makeatletter
\tikzset{join/.code=\tikzset{after node path={%
			\ifx\tikzchainprevious\pgfutil@empty\else(\tikzchainprevious)%
			edge[every join]#1(\tikzchaincurrent)\fi}}}
\makeatother
%
\tikzset{>=stealth',every on chain/.append style={join},
	every join/.style={->}}
\tikzstyle{labeled}=[execute at begin node=$\scriptstyle,
execute at end node=$]

\usetikzlibrary{arrows,positioning,shapes.geometric}

%

\usepackage[latin1]{inputenc}
\usepackage{times}
\usepackage{tikz}
\setbeamertemplate{caption}{\raggedright\insertcaption\par}


\AtBeginSection[]{

}
%----------------------------------------------------------------------------------------
%	TITLE PAGE
%----------------------------------------------------------------------------------------

\title[Self-Exciting Spatio-Temporal Models for Count Data]{Self-Exciting Spatio-Temporal Models for Count Data} % The short title appears at the bottom of every slide, the full title is only on the title page

\author{Nicholas Clark} % Your name
\institute[ISU] % Your institution as it will appear on the bottom of every slide, may be shorthand to save space
{
Iowa State University \\ % Your institution for the title page
\medskip
\textit{nclark1@iastate.edu} % Your email address
}
\date{\today} % Date, can be changed to a custom date

\begin{document}

\begin{frame}
\titlepage % Print the title page as the first slide
\end{frame}

\begin{frame}
\frametitle{Overview} % Table of contents slide, comment this block out to remove it
\tableofcontents % Throughout your presentation, if you choose to use \section{} and \subsection{} commands, these will automatically be printed on this slide as an overview of your presentation
\end{frame}

%----------------------------------------------------------------------------------------
%	PRESENTATION SLIDES
%----------------------------------------------------------------------------------------

\section{Evolution of Violence in Space and Time}

\begin{frame}
	\frametitle{Motivation: The Evolution of Violence in Space and Time}
\textit{At present, the most under-researched area of spatial criminology is that of spatio-temporal crime patterns...  the temporal component of the underlying crime distributions has languished as a largely ignored area of study} - \textbf{Crime mapping: Spatial and Temporal Challenges}, Ratcliffe (2010)

\end{frame}

\begin{frame}
	\frametitle{The Spread of Violence in Iraq 2004}
	
	\begin{figure}[ht]
		\centering
		\includegraphics[width=8cm]{Iraq2}
		\label{fig:SCSEMExample}
	\end{figure}

\end{frame}		

\begin{frame}
	\frametitle{Burglaries South Side of Chicago}
	\begin{figure}[ht]
		\centering
		\includegraphics[width=8cm]{ChiBurg}
		\label{fig:SCSEMExample}
	\end{figure}
\end{frame}		

\begin{frame}
	\frametitle{A Statistical Model for Crime (Short et al. 2008)}
		\begin{itemize}
			\item Let $s_i \in \{s_1,\cdots,s_n\}$ be fixed regions in $\mathbb{R}^2$ and $t \in \{1,\cdots,T\}$ be discrete time
			\item  Define $A(s_i,t)\equiv A(s_i,0)+B(s_i,t)$ as attractiveness, $Z(s_i,t)$ as number of observed burglaries from $(t-\Delta t,t)$
			\begin{equation}
			B(s_i,t)=\left(1-\kappa \Delta t\right) B(s_i,t- \Delta t)+\eta Z(s_i,t)
			\end{equation}
			\item Probability of occurrence at each time interval, $(t,t+\Delta t)$ is Poisson with rate, $A(s_i,t)$
		\end{itemize}
\end{frame}

 \begin{frame}
 	\frametitle{Relationship to INGARCH Model}
 	Integer Auto-Regressive Conditionally Heteroskedastic, INGARCH (1,1), or Poisson Auto-Regression Model
 	\begin{align}
 	Z(s_i,t) & \sim \mbox{Pois }(\lambda(s_i,t))\\
 	\lambda(s_i,t)&= d+ a\lambda(s_i,t-1)+b Z(s_i,t-1)
 	\end{align}
 	\begin{itemize}
 	\item Unlike GARCH, not solely a variance property
	 \item Short model is INGARCH(1,1) with $d=A(s_i,0)$, $a=(1-\kappa \delta t)$, and $b=\eta$
	 \end{itemize}
 \end{frame}

\begin{frame}
	\frametitle{Relationship to Self-Exciting Models}
	\begin{itemize}
		\item Point process introduced by Alan Hawkes with intensity
		\begin{align}
		& \lambda(t)= \nu(t) + \int_{0}^{t} g (t-u) N(ds)
		\end{align}
		\item Commonly discretized as
			\begin{align}
			& Z(s_i,t)\sim Pois(\lambda(s_i,t))\\
			& \lambda(s_i,t) = \nu + \sum_{j < t} \eta^{t-j}Z(s_i,t-j)
			\end{align}
		\item $\eta$ essentially determines $p$ in INGARCH$(0,p)$, $\eta<.5$ for stability
	\end{itemize}
\end{frame}


\begin{frame}
		\frametitle{Structural Diagram - INGARCH(1,1)}
	\begin{figure}
		\centering
		
		\begin{tikzpicture}[>=latex']
		\tikzset{block/.style= { rectangle, align=left,minimum width=.2cm,minimum height=.1cm},
			rblock/.style={draw, shape=rectangle,rounded corners=1.5em,align=center,minimum width=.2cm,minimum height=.1cm},
			input/.style={ % requires library shapes.geometric
				draw,
				trapezium,
				trapezium left angle=60,
				trapezium right angle=120,
				minimum width=2cm,
				align=center,
				minimum height=1cm
			},
		}
		\node [block]  (Y1) {\footnotesize $d$};
		\node [block, right = 2.5cm of Y1] (Y2) {\footnotesize $d$};
		\node [block, right = 2.5cm of Y2] (Y3) {\footnotesize $d$};
		\node [block, below = .5cm of Y1] (lam1) {\footnotesize $\lambda(s_1,1)$};
		\node [block, below = .5cm of Y2] (lam2) {\footnotesize $\lambda(s_2,1)$};
		\node [block, below = .5cm of Y3] (lam3) {\footnotesize $\lambda(s_3,1)$};
		\node [block, below = .5cm of lam1] (Z1) {\footnotesize $Z(s_1,1)$};
		\node [block, below = .5cm of lam2] (Z2) {\footnotesize $Z(s_2,1)$};
		\node [block, below = .5cm of lam3] (Z3) {\footnotesize $Z(s_3,1)$};
		\node [block, below = .5cm of Z1] (Z21) {\footnotesize $Z(s_1,2)$};
		\node [block, below = .5cm of Z2] (Z22) {\footnotesize $Z(s_2,2)$};
		\node [block, below = .5cm of Z3] (Z23) {\footnotesize $Z(s_3,2)$};
		\node [block, below = .5cm of Z21] (lam21) {\footnotesize $\lambda(s_1,2)$};
		\node [block, below = .5cm of Z22] (lam22) {\footnotesize $\lambda(s_2,2)$};
		\node [block, below = .5cm of Z23] (lam23) {\footnotesize $\lambda(s_3,2)$};
		\node [block, below = .5cm of lam21] (Y21) {\footnotesize $d$};
		\node [block, below = .5cm of lam22] (Y22) {\footnotesize $d$};
		\node [block, below = .5cm of lam23] (Y23) {\footnotesize $d$};
		\draw [->] (Y1) edge[line width=.4mm] (lam1);
		\draw [->] (Y2) edge[line width=.4mm] (lam2);
		\draw [->] (Y3) edge[line width=.4mm] (lam3);
		\draw [->] (lam1) edge[line width=.4mm] (Z1);
		\draw [->] (lam2) edge[line width=.4mm] (Z2);
		\draw [->] (lam3) edge[line width=.4mm] (Z3);
		\draw [->] (Y21) edge[line width=.4mm] (lam21);
		\draw [->] (Y22) edge[line width=.4mm] (lam22);
		\draw [->] (Y23) edge[line width=.4mm] (lam23);
		\draw [->] (lam21) edge[line width=.4mm] (Z21);
		\draw [->] (lam22) edge[line width=.4mm] (Z22);
		\draw [->] (lam23) edge[line width=.4mm] (Z23);
			\node[label={[red]above:{Time Period 1}},draw=red, fit=(Y1) (Y2) (Y3) (lam1) (lam2) (lam3) (Z1) (Z2) (Z3)](Fit1) {};
			\node[label={[blue]below:{Time Period 2}},draw=blue, fit=(Y21) (Y22) (Y23) (lam21) (lam22) (lam23) (Z21) (Z22) (Z23)](Fit1) {};
		\draw[->] (lam1) edge[bend right=50,line width=.5mm] (lam21);
		\draw[->] (lam2) edge[bend right=50,line width=.5mm] (lam22);
		\draw[->] (lam3) edge[bend right=50,line width=.5mm] (lam23);
		\draw[->] (Z1) edge[bend left=50,line width=.5mm] (lam21);
		\draw[->] (Z2) edge[bend left=50,line width=.5mm] (lam22);
		\draw[->] (Z3) edge[bend left=50,line width=.5mm] (lam23);
	
		\end{tikzpicture}
	\end{figure}
\end{frame}

 \begin{frame}
 	\frametitle{\small Spatially Correlated Self-Exciting Model (Clark \& Dixon, 2017)}
 	$Y(s_i,t)$ - Spatially Correlated Latent Gaussian
 	
 	\begin{figure}
 		\centering
 		
 		\begin{tikzpicture}[>=latex']
 		\tikzset{block/.style= { rectangle, align=left,minimum width=.2cm,minimum height=.1cm},
 			rblock/.style={draw, shape=rectangle,rounded corners=1.5em,align=center,minimum width=.2cm,minimum height=.1cm},
 			input/.style={ % requires library shapes.geometric
 				draw,
 				trapezium,
 				trapezium left angle=60,
 				trapezium right angle=120,
 				minimum width=2cm,
 				align=center,
 				minimum height=1cm
 			},
 		}
 		\node [block]  (Y1) {\footnotesize $\exp(Y(s_1,1))$};
 		\node [block, right = .5cm of Y1] (Y2) {\footnotesize $\exp(Y(s_2,1))$};
 		\node [block, right = .5cm of Y2] (Y3) {\footnotesize $\exp(Y(s_3,1))$};
 		\node [block, below = .5cm of Y1] (lam1) {\footnotesize $\lambda(s_1,1)$};
 		\node [block, below = .5cm of Y2] (lam2) {\footnotesize $\lambda(s_2,1)$};
 		\node [block, below = .5cm of Y3] (lam3) {\footnotesize $\lambda(s_3,1)$};
 		\node [block, below = .5cm of lam1] (Z1) {\footnotesize $Z(s_1,1)$};
 		\node [block, below = .5cm of lam2] (Z2) {\footnotesize $Z(s_2,1)$};
 		\node [block, below = .5cm of lam3] (Z3) {\footnotesize $Z(s_3,1)$};
 		\node [block, below = .5cm of Z1] (Z21) {\footnotesize $Z(s_1,2)$};
 		\node [block, below = .5cm of Z2] (Z22) {\footnotesize $Z(s_2,2)$};
 		\node [block, below = .5cm of Z3] (Z23) {\footnotesize $Z(s_3,2)$};
 		\node [block, below = .5cm of Z21] (lam21) {\footnotesize $\lambda(s_1,2)$};
 		\node [block, below = .5cm of Z22] (lam22) {\footnotesize $\lambda(s_2,2)$};
 		\node [block, below = .5cm of Z23] (lam23) {\footnotesize $\lambda(s_3,2)$};
 		\node [block, below = .5cm of lam21] (Y21) {\footnotesize $\exp(Y(s_1,2))$};
 		\node [block, below = .5cm of lam22] (Y22) {\footnotesize $\exp(Y(s_2,2))$};
 		\node [block, below = .5cm of lam23] (Y23) {\footnotesize $\exp(Y(s_3,2))$};
 		\draw [<->] (Y1) edge[line width=.4mm] (Y2);
 		\draw [<->] (Y2) edge[line width=.4mm] (Y3);
 		\draw [->] (Y1) edge[line width=.4mm] (lam1);
 		\draw [->] (Y2) edge[line width=.4mm] (lam2);
 		\draw [->] (Y3) edge[line width=.4mm] (lam3);
 		\draw [->] (lam1) edge[line width=.4mm] (Z1);
 		\draw [->] (lam2) edge[line width=.4mm] (Z2);
 		\draw [->] (lam3) edge[line width=.4mm] (Z3);
 		\draw [<->] (Y21) edge[line width=.4mm] (Y22);
 		\draw [<->] (Y22) edge[line width=.4mm] (Y23);
 		\draw [->] (Y21) edge[line width=.4mm] (lam21);
 		\draw [->] (Y22) edge[line width=.4mm] (lam22);
 		\draw [->] (Y23) edge[line width=.4mm] (lam23);
 		\draw [->] (lam21) edge[line width=.4mm] (Z21);
 		\draw [->] (lam22) edge[line width=.4mm] (Z22);
 		\draw [->] (lam23) edge[line width=.4mm] (Z23);
 		
 		\node[label={[red]above:{Time Period 1}},draw=red, fit=(Y1) (Y2) (Y3) (lam1) (lam2) (lam3) (Z1) (Z2) (Z3)](Fit1) {};
 		\node[label={[blue]below:{Time Period 2}},draw=blue, fit=(Y21) (Y22) (Y23) (lam21) (lam22) (lam23) (Z21) (Z22) (Z23)](Fit1) {};
 		\draw[->] (Z1) edge[bend right=50,line width=.5mm] (lam21);
 		\draw[->] (Z2) edge[bend right=50,line width=.5mm] (lam22);
 		\draw[->] (Z3) edge[bend right=50,line width=.5mm] (lam23);
 		\end{tikzpicture}
 		\caption{Structure Diagram for statistical model given in \eqref{eq:example}} \label{fig:structure1}
 	\end{figure}
 \end{frame}
\begin{frame}
	\frametitle{\small Spatially Correlated Self-Exciting Model (Clark \& Dixon, 2017)}
	\begin{itemize}
		\item \textbf{Theory:} Exist common cause between geographically similar locations, regions that experience uptick in violence likely to have short term self-excitation
		\pause
		\item Mixture of two processes that influence expectation : LGCP and Hawkes process
		\item Hawkes process letting $g(t-j)=\eta$ if $(t-j)=1$, $0$ otherwise
		\begin{align}
		& Z(\boldsymbol{s_i},t)|\lambda(\boldsymbol{s_i},t) \sim \text{Pois }(\mu(\boldsymbol{s_i},t)) \label{eq:Full Model2}\\
		& \lambda(\boldsymbol{s_i},t) = \exp(Y(\boldsymbol{s_i},t)) + \eta Z(\boldsymbol{s_i},t-1) \nonumber \\
		& Y(\boldsymbol{s_i},t) = \theta_1 \sum_{\boldsymbol{s_j}\in N(\boldsymbol{s_i})}Y(\boldsymbol{s_j},t) + \epsilon(\boldsymbol{s_i},t) \nonumber\\
		&\epsilon(\boldsymbol{s_i},t) \sim Gau(0,\sigma^2) \nonumber
		\end{align}
		
		
	\end{itemize}
\end{frame}
%------------------------------------------------
% Sections can be created in order to organize your presentation into discrete blocks, all sections and subsections are automatically printed in the table of contents as an overview of the talk
%------------------------------------------------

 % A subsection can be created just before a set of slides with a common theme to further break down your presentation into chunks
 \begin{frame}
 	\frametitle{\small Spatially Correlated Self-Exciting Model (Clark \& Dixon, 2017)}
 	$Y(s_i,t)$ - Spatially Correlated Latent Gaussian
 	
 	\begin{figure}
 		\centering
 		
 		\begin{tikzpicture}[>=latex']
 		\tikzset{block/.style= { rectangle, align=left,minimum width=.2cm,minimum height=.1cm},
 			rblock/.style={draw, shape=rectangle,rounded corners=1.5em,align=center,minimum width=.2cm,minimum height=.1cm},
 			input/.style={ % requires library shapes.geometric
 				draw,
 				trapezium,
 				trapezium left angle=60,
 				trapezium right angle=120,
 				minimum width=2cm,
 				align=center,
 				minimum height=1cm
 			},
 		}
 		\node [block]  (Y1) {\footnotesize $\exp(Y(s_1,1))$};
 		\node [block, right = .5cm of Y1] (Y2) {\footnotesize $\exp(Y(s_2,1))$};
 		\node [block, right = .5cm of Y2] (Y3) {\footnotesize $\exp(Y(s_3,1))$};
 		\node [block, below = .5cm of Y1] (lam1) {\footnotesize $\lambda(s_1,1)$};
 		\node [block, below = .5cm of Y2] (lam2) {\footnotesize $\lambda(s_2,1)$};
 		\node [block, below = .5cm of Y3] (lam3) {\footnotesize $\lambda(s_3,1)$};
 		\node [block, below = .5cm of lam1] (Z1) {\footnotesize $Z(s_1,1)$};
 		\node [block, below = .5cm of lam2] (Z2) {\footnotesize $Z(s_2,1)$};
 		\node [block, below = .5cm of lam3] (Z3) {\footnotesize $Z(s_3,1)$};
 		\node [block, below = .5cm of Z1] (Z21) {\footnotesize $Z(s_1,2)$};
 		\node [block, below = .5cm of Z2] (Z22) {\footnotesize $Z(s_2,2)$};
 		\node [block, below = .5cm of Z3] (Z23) {\footnotesize $Z(s_3,2)$};
 		\node [block, below = .5cm of Z21] (lam21) {\footnotesize $\lambda(s_1,2)$};
 		\node [block, below = .5cm of Z22] (lam22) {\footnotesize $\lambda(s_2,2)$};
 		\node [block, below = .5cm of Z23] (lam23) {\footnotesize $\lambda(s_3,2)$};
 		\node [block, below = .5cm of lam21] (Y21) {\footnotesize $\exp(Y(s_1,2))$};
 		\node [block, below = .5cm of lam22] (Y22) {\footnotesize $\exp(Y(s_2,2))$};
 		\node [block, below = .5cm of lam23] (Y23) {\footnotesize $\exp(Y(s_3,2))$};
 		\draw [<->] (Y1) edge[line width=.4mm] (Y2);
 		\draw [<->] (Y2) edge[line width=.4mm] (Y3);
 		\draw [->] (Y1) edge[line width=.4mm] (lam1);
 		\draw [->] (Y2) edge[line width=.4mm] (lam2);
 		\draw [->] (Y3) edge[line width=.4mm] (lam3);
 		\draw [->] (lam1) edge[line width=.4mm] (Z1);
 		\draw [->] (lam2) edge[line width=.4mm] (Z2);
 		\draw [->] (lam3) edge[line width=.4mm] (Z3);
 		\draw [<->] (Y21) edge[line width=.4mm] (Y22);
 		\draw [<->] (Y22) edge[line width=.4mm] (Y23);
 		\draw [->] (Y21) edge[line width=.4mm] (lam21);
 		\draw [->] (Y22) edge[line width=.4mm] (lam22);
 		\draw [->] (Y23) edge[line width=.4mm] (lam23);
 		\draw [->] (lam21) edge[line width=.4mm] (Z21);
 		\draw [->] (lam22) edge[line width=.4mm] (Z22);
 		\draw [->] (lam23) edge[line width=.4mm] (Z23);
 	
 		\node[label={[red]above:{Time Period 1}},draw=red, fit=(Y1) (Y2) (Y3) (lam1) (lam2) (lam3) (Z1) (Z2) (Z3)](Fit1) {};
 		\node[label={[blue]below:{Time Period 2}},draw=blue, fit=(Y21) (Y22) (Y23) (lam21) (lam22) (lam23) (Z21) (Z22) (Z23)](Fit1) {};
 			\draw[->] (Z1) edge[bend right=50,line width=.5mm] (lam21);
 			\draw[->] (Z2) edge[bend right=50,line width=.5mm] (lam22);
 			\draw[->] (Z3) edge[bend right=50,line width=.5mm] (lam23);
 		\end{tikzpicture}
 		\caption{Structure Diagram for statistical model given in \eqref{eq:example}} \label{fig:structure1}
 	\end{figure}
 \end{frame}
 \section{SPINGARCH Model} 
 \begin{frame}
 	\frametitle{SPINGARCH(1,1) Model}
 	Spatially Correlated INGARCH(1,1) Model
 	
 	\begin{figure}
 		\centering
 		
 		\begin{tikzpicture}[>=latex']
 		\tikzset{block/.style= { rectangle, align=left,minimum width=.2cm,minimum height=.1cm},
 			rblock/.style={draw, shape=rectangle,rounded corners=1.5em,align=center,minimum width=.2cm,minimum height=.1cm},
 			input/.style={ % requires library shapes.geometric
 				draw,
 				trapezium,
 				trapezium left angle=60,
 				trapezium right angle=120,
 				minimum width=2cm,
 				align=center,
 				minimum height=1cm
 			},
 		}
 		\node [block]  (Y1) {\footnotesize $\exp(Y(s_1,1))$};
 		\node [block, right = .5cm of Y1] (Y2) {\footnotesize $\exp(Y(s_2,1))$};
 		\node [block, right = .5cm of Y2] (Y3) {\footnotesize $\exp(Y(s_3,1))$};
 		\node [block, below = .5cm of Y1] (lam1) {\footnotesize $\lambda(s_1,1)$};
 		\node [block, below = .5cm of Y2] (lam2) {\footnotesize $\lambda(s_2,1)$};
 		\node [block, below = .5cm of Y3] (lam3) {\footnotesize $\lambda(s_3,1)$};
 		\node [block, below = .5cm of lam1] (Z1) {\footnotesize $Z(s_1,1)$};
 		\node [block, below = .5cm of lam2] (Z2) {\footnotesize $Z(s_2,1)$};
 		\node [block, below = .5cm of lam3] (Z3) {\footnotesize $Z(s_3,1)$};
 		\node [block, below = .5cm of Z1] (Z21) {\footnotesize $Z(s_1,2)$};
 		\node [block, below = .5cm of Z2] (Z22) {\footnotesize $Z(s_2,2)$};
 		\node [block, below = .5cm of Z3] (Z23) {\footnotesize $Z(s_3,2)$};
 		\node [block, below = .5cm of Z21] (lam21) {\footnotesize $\lambda(s_1,2)$};
 		\node [block, below = .5cm of Z22] (lam22) {\footnotesize $\lambda(s_2,2)$};
 		\node [block, below = .5cm of Z23] (lam23) {\footnotesize $\lambda(s_3,2)$};
 		\node [block, below = .5cm of lam21] (Y21) {\footnotesize $\exp(Y(s_1,2))$};
 		\node [block, below = .5cm of lam22] (Y22) {\footnotesize $\exp(Y(s_2,2))$};
 		\node [block, below = .5cm of lam23] (Y23) {\footnotesize $\exp(Y(s_3,2))$};
 		\draw [<->] (Y1) edge[line width=.4mm] (Y2);
 		\draw [<->] (Y2) edge[line width=.4mm] (Y3);
 		\draw [->] (Y1) edge[line width=.4mm] (lam1);
 		\draw [->] (Y2) edge[line width=.4mm] (lam2);
 		\draw [->] (Y3) edge[line width=.4mm] (lam3);
 		\draw [->] (lam1) edge[line width=.4mm] (Z1);
 		\draw [->] (lam2) edge[line width=.4mm] (Z2);
 		\draw [->] (lam3) edge[line width=.4mm] (Z3);
 		\draw [<->] (Y21) edge[line width=.4mm] (Y22);
 		\draw [<->] (Y22) edge[line width=.4mm] (Y23);
 		\draw [->] (Y21) edge[line width=.4mm] (lam21);
 		\draw [->] (Y22) edge[line width=.4mm] (lam22);
 		\draw [->] (Y23) edge[line width=.4mm] (lam23);
 		\draw [->] (lam21) edge[line width=.4mm] (Z21);
 		\draw [->] (lam22) edge[line width=.4mm] (Z22);
 		\draw [->] (lam23) edge[line width=.4mm] (Z23);
 		\node[label={[red]above:{Time Period 1}},draw=red, fit=(Y1) (Y2) (Y3) (lam1) (lam2) (lam3) (Z1) (Z2) (Z3)](Fit1) {};
 		\node[label={[blue]below:{Time Period 2}},draw=blue, fit=(Y21) (Y22) (Y23) (lam21) (lam22) (lam23) (Z21) (Z22) (Z23)](Fit1) {};
 			\draw[->] (lam1) edge[bend right=50,line width=.5mm] (lam21);
 			\draw[->] (lam2) edge[bend right=50,line width=.5mm] (lam22);
 			\draw[->] (lam3) edge[bend right=50,line width=.5mm] (lam23);
 			\draw[->] (Z1) edge[bend left=50,line width=.5mm] (lam21);
 			\draw[->] (Z2) edge[bend left=50,line width=.5mm] (lam22);
 			\draw[->] (Z3) edge[bend left=50,line width=.5mm] (lam23);
 		\end{tikzpicture}
 		\end{figure}
 \end{frame}
 
  \begin{frame}
  	\frametitle{SPINGARCH Model}
  	\begin{itemize}
  	\item $Z(s_i,t)|Y(s_i,t),\mathcal{H}_{Z(s_i)}\sim\mbox{Pois }(\lambda(s_i,t))$ where $\mathcal{H}_{Z(s_i)}$ is history of process at location $s_i$
  	\begin{align}
	\lambda(s_i,t)&=\exp \left[ Y(s_i,t) \right] + \eta Z(s_i,t-1) + \kappa E\left[Z(s_i,t-1)\right]\nonumber\\
  	\end{align}
\item Define: $N_i=\{s_j :s_j\text{ is a spatial neighbor of } s_i\}$
  	\begin{align}
  	& Y(s_i,t)|\boldsymbol{Y}(N_i)\sim N(\mu(s_i,t),\sigma_{sp}^2) \label{eq:Latent Dependency}\\
  	& \mu(s_i,t) = \alpha(s_i)+ \zeta \sum_{s_j \in N_i} \{Y(s_j,t)-\alpha(s_j)\} \nonumber.
  	\end{align}
  	\pause
  	\item $\eta=0,\kappa=0$ Poisson - CAR, $\sigma^2_{sp} \to 0$, INGARCH(1,1)/Short model
  	\end{itemize}
  \end{frame}
  
  
    \begin{frame}
    	\frametitle{SPINGARCH Model - Parameter Space}
    	\begin{itemize}
    		\item $Z(s_i,t)|Y(s_i,t),\mathcal{H}_{Z(s_i)}\sim\mbox{Pois }(\lambda(s_i,t))$ where $\mathcal{H}_{Z(s_i)}$ is history of process at location $s_i$
    		\begin{align}
    		\lambda(s_i,t)&=\exp \left[ Y(s_i,t) \right] + \eta Z(s_i,t-1) + \kappa E\left[Z(s_i,t-1)\right]\nonumber\\
    		& Y(s_i,t)|\boldsymbol{Y}(N_i)\sim N(\mu(s_i,t),\sigma_{sp}^2) \nonumber\\
    		& \mu(s_i,t) = \alpha(s_i)+ \zeta \sum_{s_j \in N_i} \{Y(s_j,t)-\alpha(s_j)\} \nonumber.
    		\end{align}
    		\item $\zeta \in  (\psi_{(1)}^{-1},\psi_{(n)}^{-1})$ where $\psi_{(i)}$ is the $i$th largest eigenvalue of $\sigma_{sp}^2 \boldsymbol{N}$ where $N$ is adjacency matrix
    		\item For stationarity, $\eta>0$, $\kappa>0$, $\eta+\kappa <1$
  
    	\end{itemize}
    \end{frame}
    
  
  \begin{frame}
  	\frametitle{Data Realizations}
  	50 Spatial Observations on $\mathbb{R}^1$, 100 Temporal Observations
\begin{figure}[t!]
	\centering
	\includegraphics[width=0.475\textwidth]{SPINGARCHPlot}
	\hfill
	\includegraphics[width=0.475\textwidth]{SingePlot}
\end{figure}
 \begin{align}
 & \lambda(s_i,t) =\exp \left[ Y(s_i,t) \right] + 0.1 \mbox{ Z}(s_i,t-1)+ \mbox{0.4 } \lambda(s_i,t-1)\nonumber\\
 & Y(s_i,t)|\boldsymbol{Y}(N_i)  \sim \mbox{Gau}(\mu(s_i,t),0.5)\nonumber\\ 
 & \mu(s_i,t)  = 0+ 0.49 \sum_{s_j \in N_i} \{Y(s_j,t)\}\label{eq:SEPoissonGen}.
 \end{align}
  \end{frame}
  
  
    \begin{frame}
    	\frametitle{SPINGARCH Model as Markov Chain}
    	\begin{itemize}
    	\item Let $\boldsymbol{\lambda_t}=\left(\lambda(s_1,t),\lambda(s_2,t),\cdots,\lambda(s_{n_d},t)\right)^T$
    	
  \begin{align}
  & Z(s_i,t)|\lambda(s_i,t) \sim \mbox{Pois}(\lambda(s_i,t)) \label{eq:timeseries2} \nonumber\\
  & E[Z(s_i,t)]=\lambda(s_i,t)\nonumber\\
  & \boldsymbol{\lambda_t} = \exp(\boldsymbol{Y_t})+\eta \boldsymbol{Z_{t-1}}+\kappa \boldsymbol{\lambda_{t-1}}\nonumber\\
  & \boldsymbol{Y_t} \sim \mbox{Gau} (\boldsymbol{\alpha_t},(I_{{n_d},{n_d}}-\boldsymbol{C})^{-1}\boldsymbol{M})
  \end{align}
  \item Markov chain for $\boldsymbol{\lambda_t}$ on State space, $(\mathbb{R}^{+})^{n_d}$
  \end{itemize}
    \end{frame}
    
    \begin{frame}
    	\frametitle{Impact of Initial Condtions and Recursion}
    	By recursion
    	\begin{align*}
    	& [\lambda(s_i,t)|\lambda(s_i,0)=B] = \exp(Y(s_i,t))+\kappa \lambda(s_i,t-1) + \eta Z(s_i,t-1)\\
    	 =& \exp(Y(s_i,t))+\kappa \left[\exp(Y(s_i,t-1))+\kappa \lambda(s_i,t-2)\right.\\
    	 & \left.+ \eta Z(s_i,t-2)\right] + \eta Z(s_i,t-1)\\
    	&\cdots\\
    	 =&\sum_{k=0}^{t-1} \kappa\exp(Y(s_i,t-k)) +\sum_{k=0}^{t-1} \kappa\eta Z(s_i,t-k-1)+\kappa^t B. \label{eq:Recursion}
    	\end{align*}
    \end{frame}

   
   \section{SPINGARCH Model Properties}
   

   
    \begin{frame}
    	\frametitle{Geometric Ergodicity with Finite Moments}
    		Under the parameter space restriction, $\eta,\kappa\geq0$ and $\eta+\kappa<1$, the SPINGARCH (1,1) is geometrically ergodic and admits a unique stationary distributions that has finite first two moments.
\begin{proof}[Sketch of Proof]
\textbf{Meyn and Tweedie (15.0.1)} need to show aperiodic, $\phi$-irreducible and $\exists$ test function $V(.)$ such that  $E[V(\boldsymbol{\lambda}_{t+1})|\boldsymbol{\lambda}_t=\boldsymbol{\lambda_*}]\leq \psi V(\boldsymbol{\lambda_*})+L \mbox{ I}(\boldsymbol{\lambda_*} \in C)$ holds where $\psi \in (0,1)$, $L \in (0,\infty)$ and $I(.)$ is the indicator function and $C$ is a petite set.
\newline

\textbf{Basic Idea}:  With positive probability, $\exists$ a realization $Z(s_i,1)=Z(s_i,2)=\cdots=Z(s_i,t-1) =0$.  Along that chain, $P(\lambda(s_i,t))\in A=P(\exp(Y(s_i,t))\in A-\kappa^t B)$.  If $\kappa^T B > \sup A$ run chain longer.
	\end{proof}
    \end{frame}
      \begin{frame}
      	\frametitle{Geometric Ergodicity with Finite Moments}
      	Under the parameter space restriction, $\eta,\kappa\geq0$ and $\eta+\kappa<1$, the SPINGARCH (1,1) is geometrically ergodic and admits a unique stationary distributions that has finite first two moments.
      	\begin{proof}[Sketch of Proof Cont.]
      	Test function $V(\lambda)=1+\lambda^2$ works for $E[V(\boldsymbol{\lambda}_{t+1})|\boldsymbol{\lambda}_t=\boldsymbol{\lambda_*}]\leq \psi V(\boldsymbol{\lambda_*})+L \mbox{ I}(\boldsymbol{\lambda_*} \in C)$.
      	\newline
      	
      	$\implies$ Unique stationary distribution, goes to geometrically fast.  Specific choice of $V(.)$ gives (at least) finite first two moments (can be extended likely as in Fokianos, 2009.)
      	
      	\end{proof}
      \end{frame}
    
     \begin{frame}
     	\frametitle{Comparison of Moment Properties with INGARCH Model}
     	
     	\begin{align}
     	& E[Z(s_i,t)] = \frac{d}{1-(a+b)}\\
     	& \mbox{Var}[Z(s_i,t)] = \frac{1-(a+b)^2+b^2}{1-(a+b)^2} E[Z(s_i,t)] \\
     	& \mbox{Cov} [Z(s_i,t),Z(s_i,t-h)] = \frac{b(1-a(a+b))(a+b)^h}{1-(a+b)^2} E[Z(s_i,t)]\\
     	& \mbox{Var-Mean Ratio} [Z(s_i,t)] = 1+\frac{b^2}{1-(a+b)^2}
     	\end{align}
     	
     \end{frame}
 
     \begin{frame}
     	\frametitle{Issues}
     	
     	\begin{align}
     	& \mbox{Cor} [Z(s_i,t),Z(s_i,t-1)] = \frac{b(a+b)(a^2+ab-1)}{a^2+2ab-1}\\
     	& \mbox{Var-Mean Ratio} [Z(s_i,t)] = 1+\frac{b^2}{1-(a+b)^2}\\
     	\end{align}
    For fixed Var-Mean Ratio at 2 $\implies b=1/2 (-a +\sqrt{2-a^2})$. 
    \begin{figure}[t!]
    	\centering
    	\includegraphics[width=0.4\textwidth]{INGARCHIssue}
    \end{figure}
     \end{frame}
     
       \begin{frame}
       	\frametitle{Increased Modeling Flexibility with SPINGARCH(1,1)}
       	Define $\Sigma_{i,j}$ as $i,j$ entry of $I_{{n_d},{n_d}}-\boldsymbol{C})^{-1}\boldsymbol{M}$
     \begin{align}
     & E\left[Z(s_i,t)\right]=\frac{1}{1-\eta-\kappa}\exp(\alpha+\frac{\Sigma_{1,1}}{2})\\
     &\small \mbox{Var }(Z(s_i,t))=\frac{1}{1-(\kappa+\eta)^2} \mbox{Var }(\exp(Y(s_i,t)))+\frac{1-\kappa^2-2\kappa \eta}{1-(\kappa+\eta)^2}E(Z(s_i,t))
    \end{align}
       \end{frame}
     

 \section{Bayesian Inference}
   \begin{frame}
   	\frametitle{Difficulties and Limitations}
   \end{frame}
 
 \begin{frame}
 	\frametitle{Efficient Bayesian Inference}
 \end{frame}
 

  \section{Simulation}
\begin{frame}
	\frametitle{Simulation and Estimation}
\end{frame}

\begin{frame}
	\frametitle{Replication of Second Order Properties}
\end{frame}
 
  \section{Burglaries in South Side of Chicago}
 \begin{frame}
 	\frametitle{Data and Assumptions}
 \end{frame}
 
  \begin{frame}
  	\frametitle{INGARCH(1,1) Model - Motivated by Short}
  \end{frame}
  
   \begin{frame}
   	\frametitle{SPINGARCH(1,1) Model}
   \end{frame}
 
  \begin{frame}
  	\frametitle{Model Assessment}
  \end{frame}
 
 \section{Difficulties with Off the Shelf Inference}
 
   \begin{frame}
   	\frametitle{SPINGARCH (0,1) - Spatially Correlated Self-Exciting Model (Clark \& Dixon, \textit{in press})}
   \end{frame}
 
  \begin{frame}
  	\frametitle{Laplace Approximation Motivated by INLA}
  \end{frame}
  
  \begin{frame}
    	\frametitle{Comparison to Stan}
  \end{frame}
  
  \begin{frame}
  	\frametitle{Limitations and Bias}
  \end{frame}


 \section{Extended Laplace Approximation}
  \begin{frame}
  	\frametitle{Laplace Approximation Issues for Space Time}
  \end{frame}
  
   \begin{frame}
   	\frametitle{An Extended Laplace Approximation}
   \end{frame}
   
    \begin{frame}
    	\frametitle{Comparison}
    \end{frame}

  
  \begin{frame}
  	\frametitle{Violent Crimes in Chicago}
  	\end{frame}
  	
  	
  	\begin{frame}
  		\frametitle{Comparison of Inferential Methodologies}
  	\end{frame}
  	
   \section{Summary/Way Ahead}
  		\begin{frame}
  			\frametitle{Summary/Way Ahead}
  		\end{frame}
\appendix

\section{Post Talk Slides} 

\begin{frame}
	\frametitle{Contributions}
\end{frame}

\begin{frame}
	\frametitle{Remaining Gaps}
\end{frame}

\begin{frame}
	\frametitle{Future Research}
\end{frame}

\begin{frame}
	\frametitle{Achievements}
\end{frame}

\begin{frame}
	\frametitle{Self-Reflection}
\end{frame}

\begin{frame}
	\frametitle{Issues}
	\begin{itemize}
		\item 155 Iraqi districts, 96 months $\implies$ 14880 x 14880 precision matrix
		\item Sparse, but not overly so
		\begin{itemize}
			\item Out of $\approx$ 250 million entries, $\approx$ 150 thousand are non-zero for SCSEM and $\approx$ 700 thousand are non-zero for RDSEM
		\end{itemize}
		\item For Bayesian inference MCMC is gold standard as errors can be driven to zero - Oftentimes not practical due to time constraints
		\item In standard geospatial models, INLA has emerged as alternate technique due to Rue et al 2009
		\begin{itemize}
			\item R-INLA, ``off the shelf", models for Besag-York-Molle, Markov Random Field (MRF) approximation to Matern, AR(1), and others
		\end{itemize}
		\item Accuracy vs Speed/computation
	\end{itemize}
\end{frame}

\begin{frame}
	\frametitle{Connection to Laplace Approximation}
	\begin{itemize}
		\item Evaluated at $x^*(\boldsymbol{\theta})$, this becomes:
		\begin{equation}
		\tilde{\pi}(\boldsymbol{\theta}|\boldsymbol{y}) \propto \frac{\pi (\boldsymbol{x}^*(\boldsymbol{\theta}),\boldsymbol{\theta},\boldsymbol{y})}{|Prec^*(\boldsymbol{\theta})/2\pi|^{1/2}}
		\end{equation}
		\item Compare to numerator of Tierney and Kadane (1986):
		\begin{equation}
		\left(\frac{\det \Delta_h(\hat{\theta}_2)}{2\pi n \det \Delta_h \hat{\theta}}\right)^{1/2}\frac{\pi(\theta_1,\hat{\theta}_2)\exp\left(\ell (y| \theta_1,\hat{\theta_2})\right)} {\pi(\hat{\boldsymbol{\theta}})\exp\left(\ell (y|\hat{\boldsymbol{\theta}})\right)} 
		\end{equation}
		\item The approximation error is $O(n^{-3/2})$ on fixed lattice
	\end{itemize}
\end{frame}


\begin{frame}
	\frametitle{Marginals of Diffusion Parameters}
	\begin{itemize}
		\item Numerically approximated Hessian of previous step could be used
		\item More generally
		\[
		\pi(\theta_j|\boldsymbol{Y}) \sim
		\begin{cases} 
		\hfill Gaus(\hat{\theta_j},\sigma^2_+)   \hfill & \text{ for} \theta_j > \hat{\theta_j} \\
		\hfill Gaus(\hat{\theta_j},\sigma^2_-)   \hfill & \text{ for} \theta_j < \hat{\theta_j} \\
		\end{cases}
		\]
		
		\begin{block}{Lemma (Martins et al, 2013)}
			Assume $(\theta_1,\theta_2,\theta_3)^T \sim Gaus(\boldsymbol{0},\Sigma)$
			\begin{equation}
				\frac{1}{2}\left(\theta_1,E(\boldsymbol{\theta_{-1}}|\theta_1)\right)\Sigma^{-1}\left(\theta_1,E(\boldsymbol{\theta_{-1}}|\theta_1)\right)^T = -\frac{1}{2}\frac{\theta_1}{\Sigma_{1,1}}
			\end{equation}
		\end{block}
		\item Joint evaluated at conditional behaves like marginal, numerically estimate $\sigma^2_+$, $\sigma^2_-$
	\end{itemize}
\end{frame}




%------------------------------------------------


%----------------------------------------------------------------------------------------

\end{document}


\begin{frame}
	\frametitle{References}
	\footnotesize{
		\begin{thebibliography}{99} % Beamer does not support BibTeX so references must be inserted manually as below
			\bibitem[Rue, Martino, Chopin]{p1} Rue, Martino, and Chopin (2009)
			\newblock Approximate Bayesian inference for latent Gaussian models by using integrated nested Laplace approximations
			\newblock \emph{Journal of the Royal Statistical Society B.} Vol 71, part 2
			\bibitem[Tierney and Kadane]{p2} Tierney, L. and Kadane, J.B. (1986)
			\newblock Accurate approximations for posterior moments and marginal densities.
			\newblock \emph{Journal of the American Statistical Association} Vol. 81, 82-86
			\bibitem[Finley, Banerjee, Gelfand]{p2} Finley, Banerjee, Gelfand (2013)
			\newblock spBayes for large univariate and multivariate point-referenced spatio-temporal data models
			\newblock \emph{arXiv preprint arXiv:1310.8192}
			
			\bibitem[Martins, Simpson, Lindgren, Rue]{p4} Martins, T., Simpson, D., Findgren, F., Rue, H. (2013)
			\newblock Bayesian computing with INLA: new features
			\newblock \emph{Computational Statistics and Data Analysis} Vol. 67, 68-83.
			
		\end{thebibliography}
	}
\end{frame}


\begin{frame}{Simulation RDSEM}
	\begin{center}
		Process Model - $\exp(X(\boldsymbol{s_i},t))$
	\end{center}
	\qquad
	\begin{tikzpicture}[>=latex']
	\tikzset{block/.style= { rectangle, align=left,minimum width=.2cm,minimum height=.1cm},
		rblock/.style={draw, shape=rectangle,rounded corners=1.5em,align=center,minimum width=.2cm,minimum height=.1cm},
		input/.style={ % requires library shapes.geometric
			draw,
			trapezium,
			trapezium left angle=60,
			trapezium right angle=120,
			minimum width=2cm,
			align=center,
			minimum height=1cm
		},
	}
	\node [block]  (x1) {\footnotesize 1.2};
	\node [block, right = .1cm of x1] (x2) {\footnotesize 3.4};
	\node [block, right = .1cm of x2] (x3) {\footnotesize .8};
	\node [block, right = .1cm of x3] (x4) {\footnotesize 1.5};
	\node [block, below = .1cm of x1] (y1) {\footnotesize 4.6};
	\node [block, below = .1cm of x2] (y2) {\footnotesize 3.4};
	\node [block, below = .1cm of x3] (y3) {\footnotesize 2.9};
	\node [block, below = .1cm of x4] (y4) {\footnotesize 1.5};
	\node [block, below = .1cm of y1] (z1) {\footnotesize 2.4};
	\node [block, below = .1cm of y2] (z2) {\footnotesize 1.5};
	\node [block, below = .1cm of y3] (z3) {\footnotesize 1.6};
	\node [block, below = .1cm of y4] (z4) {\footnotesize 4.0};
	\node [block, below = .1cm of z1] (q1) {\footnotesize .1};
	\node [block, below = .1cm of z2] (q2) {\footnotesize 1.4};
	\node [block, below = .1cm of z3] (q3) {\footnotesize .5};
	\node [block, below = .1cm of z4] (q4) {\footnotesize .1};
	\node [block, above = .1cm of x2] {\footnotesize Time 1};
	\end{tikzpicture}
	\quad
	\begin{tikzpicture}[>=latex']
	\tikzset{block/.style= { rectangle, align=left,minimum width=.2cm,minimum height=.1cm},
		rblock/.style={draw, shape=rectangle,rounded corners=1.5em,align=center,minimum width=.2cm,minimum height=.1cm},
		input/.style={ % requires library shapes.geometric
			draw,
			trapezium,
			trapezium left angle=60,
			trapezium right angle=120,
			minimum width=2cm,
			align=center,
			minimum height=1cm
		},
	}
	\node [block]  (x1) {\footnotesize 1.6};
	\node [block, right = .1cm of x1] (x2) {\footnotesize .7};
	\node [block, right = .1cm of x2] (x3) {\footnotesize .7};
	\node [block, right = .1cm of x3] (x4) {\footnotesize 1.5};
	\node [block, below = .1cm of x1] (y1) {\footnotesize 4.0};
	\node [block, below = .1cm of x2] (y2) {\footnotesize 42.2};
	\node [block, below = .1cm of x3] (y3) {\footnotesize 3.2};
	\node [block, below = .1cm of x4] (y4) {\footnotesize 7.7};
	\node [block, below = .1cm of y1] (z1) {\footnotesize 1.5};
	\node [block, below = .1cm of y2] (z2) {\footnotesize .6};
	\node [block, below = .1cm of y3] (z3) {\footnotesize 1.4};
	\node [block, below = .1cm of y4] (z4) {\footnotesize 1.1};
	\node [block, below = .1cm of z1] (q1) {\footnotesize .6};
	\node [block, below = .1cm of z2] (q2) {\footnotesize .5};
	\node [block, below = .1cm of z3] (q3) {\footnotesize .1};
	\node [block, below = .1cm of z4] (q4) {\footnotesize .6};
	\node [block, above = .1cm of x2] {\footnotesize Time 2};
	\end{tikzpicture}
	\quad
	\begin{tikzpicture}[>=latex']
	\tikzset{block/.style= { rectangle, align=center,minimum width=.2cm,minimum height=.1cm},
		rblock/.style={draw, shape=rectangle,rounded corners=1.5em,align=center,minimum width=.2cm,minimum height=.1cm},
		input/.style={ % requires library shapes.geometric
			draw,
			trapezium,
			trapezium left angle=60,
			trapezium right angle=120,
			minimum width=2cm,
			align=center,
			minimum height=1cm
		},
	}
	\node [block]  (x1) {\footnotesize 2.6};
	\node [block, right = .1cm of x1] (x2) {\footnotesize 1.5};
	\node [block, right = .1cm of x2] (x3) {\footnotesize 14.9};
	\node [block, right = .1cm of x3] (x4) {\footnotesize .82};
	\node [block, below = .1cm of x1] (y1) {\footnotesize 20.8};
	\node [block, below = .1cm of x2] (y2) {\footnotesize 3.27};
	\node [block, below = .1cm of x3] (y3) {\footnotesize 2.4};
	\node [block, below = .1cm of x4] (y4) {\footnotesize 5.2};
	\node [block, below = .1cm of y1] (z1) {\footnotesize 1.8};
	\node [block, below = .1cm of y2] (z2) {\footnotesize 1.7};
	\node [block, below = .1cm of y3] (z3) {\footnotesize .3};
	\node [block, below = .1cm of y4] (z4) {\footnotesize 3.7};
	\node [block, below = .1cm of z1] (q1) {\footnotesize 10.3};
	\node [block, below = .1cm of z2] (q2) {\footnotesize .2};
	\node [block, below = .1cm of z3] (q3) {\footnotesize .93};
	\node [block, below = .1cm of z4] (q4) {\footnotesize .6};
	\node [block, above = .1cm of x2] {\footnotesize Time 3};
	\end{tikzpicture}
	\begin{center}
		Realizations From Model - $Y(\boldsymbol{s_i},t)$
	\end{center}
	\qquad
	\quad
	\begin{tikzpicture}[>=latex']
	\tikzset{block/.style= { rectangle, align=left,minimum width=.2cm,minimum height=.1cm},
		rblock/.style={draw, shape=rectangle,rounded corners=1.5em,align=center,minimum width=.2cm,minimum height=.1cm},
		input/.style={ % requires library shapes.geometric
			draw,
			trapezium,
			trapezium left angle=60,
			trapezium right angle=120,
			minimum width=2cm,
			align=center,
			minimum height=1cm
		},
	}
	\node [block]  (x1) {\footnotesize 0};
	\node [block, right = .1cm of x1] (x2) {\footnotesize 6};
	\node [block, right = .1cm of x2] (x3) {\footnotesize 0};
	\node [block, right = .1cm of x3] (x4) {\footnotesize 8};
	\node [block, below = .1cm of x1] (y1) {\footnotesize 8};
	\node [block, below = .1cm of x2] (y2) {\footnotesize 4};
	\node [block, below = .1cm of x3] (y3) {\footnotesize 8};
	\node [block, below = .1cm of x4] (y4) {\footnotesize 2};
	\node [block, below = .1cm of y1] (z1) {\footnotesize 1};
	\node [block, below = .1cm of y2] (z2) {\footnotesize 5};
	\node [block, below = .1cm of y3] (z3) {\footnotesize 5};
	\node [block, below = .1cm of y4] (z4) {\footnotesize 4};
	\node [block, below = .1cm of z1] (q1) {\footnotesize 0};
	\node [block, below = .1cm of z2] (q2) {\footnotesize 0};
	\node [block, below = .1cm of z3] (q3) {\footnotesize 0};
	\node [block, below = .1cm of z4] (q4) {\footnotesize 0};
	\node [block, above = .1cm of x2] {\footnotesize Time 1};
	\end{tikzpicture}
	\qquad
	\quad
	\begin{tikzpicture}[>=latex']
	\tikzset{block/.style= { rectangle, align=left,minimum width=.2cm,minimum height=.1cm},
		rblock/.style={draw, shape=rectangle,rounded corners=1.5em,align=center,minimum width=.2cm,minimum height=.1cm},
		input/.style={ % requires library shapes.geometric
			draw,
			trapezium,
			trapezium left angle=60,
			trapezium right angle=120,
			minimum width=2cm,
			align=center,
			minimum height=1cm
		},
	}
	\node [block]  (x1) {\footnotesize 3};
	\node [block, right = .1cm of x1] (x2) {\footnotesize 6};
	\node [block, right = .1cm of x2] (x3) {\footnotesize 3};
	\node [block, right = .1cm of x3] (x4) {\footnotesize 3};
	\node [block, below = .1cm of x1] (y1) {\footnotesize 5};
	\node [block, below = .1cm of x2] (y2) {\footnotesize 44};
	\node [block, below = .1cm of x3] (y3) {\footnotesize 5};
	\node [block, below = .1cm of x4] (y4) {\footnotesize 10};
	\node [block, below = .1cm of y1] (z1) {\footnotesize 3};
	\node [block, below = .1cm of y2] (z2) {\footnotesize 1};
	\node [block, below = .1cm of y3] (z3) {\footnotesize 1};
	\node [block, below = .1cm of y4] (z4) {\footnotesize 3};
	\node [block, below = .1cm of z1] (q1) {\footnotesize 0};
	\node [block, below = .1cm of z2] (q2) {\footnotesize 1};
	\node [block, below = .1cm of z3] (q3) {\footnotesize 0};
	\node [block, below = .1cm of z4] (q4) {\footnotesize 0};
	\node [block, above = .1cm of x2] {\footnotesize Time 2};
	\end{tikzpicture}
	\qquad
	\quad
	\begin{tikzpicture}[>=latex']
	\tikzset{block/.style= { rectangle, align=center,minimum width=.2cm,minimum height=.1cm},
		rblock/.style={draw, shape=rectangle,rounded corners=1.5em,align=center,minimum width=.2cm,minimum height=.1cm},
		input/.style={ % requires library shapes.geometric
			draw,
			trapezium,
			trapezium left angle=60,
			trapezium right angle=120,
			minimum width=2cm,
			align=center,
			minimum height=1cm
		},
	}
	\node [block]  (x1) {\footnotesize 0};
	\node [block, right = .1cm of x1] (x2) {\footnotesize 6};
	\node [block, right = .1cm of x2] (x3) {\footnotesize 14};
	\node [block, right = .1cm of x3] (x4) {\footnotesize 0};
	\node [block, below = .1cm of x1] (y1) {\footnotesize 21};
	\node [block, below = .1cm of x2] (y2) {\footnotesize 13};
	\node [block, below = .1cm of x3] (y3) {\footnotesize 3};
	\node [block, below = .1cm of x4] (y4) {\footnotesize 9};
	\node [block, below = .1cm of y1] (z1) {\footnotesize 2};
	\node [block, below = .1cm of y2] (z2) {\footnotesize 1};
	\node [block, below = .1cm of y3] (z3) {\footnotesize 3};
	\node [block, below = .1cm of y4] (z4) {\footnotesize 4};
	\node [block, below = .1cm of z1] (q1) {\footnotesize 4};
	\node [block, below = .1cm of z2] (q2) {\footnotesize 0};
	\node [block, below = .1cm of z3] (q3) {\footnotesize 1};
	\node [block, below = .1cm of z4] (q4) {\footnotesize 1};
	\node [block, above = .1cm of x2] {\footnotesize Time 3};
	\end{tikzpicture}\\
\end{frame}

\begin{frame}
	\frametitle{Motivating Data - Iraq}
	\begin{itemize}
		
		\item Global Terrorism Database - Largely Media Reported, geotagged
		\item 6023 Events in Iraq from 2003-2010
		\item Aggregated Daily, geographically aggregated by data available
	\end{itemize}
	
	\begin{columns}[t]
		\column{.5\textwidth}
		\centering
		\includegraphics[width=6cm,height=4.5cm]{IZEvents}\\
		
		\column{.5\textwidth}
		\centering
		\includegraphics[width=6cm,height=4.5cm]{GTDEventsPerMonth}
	\end{columns}
\end{frame}


\begin{frame}
	\frametitle{Fit of SCSEM}
	\begin{itemize}
		\item Posterior mode at $\hat{\theta}_1=.14$, $\hat{\sigma}^2=2.62$, $\hat{\eta}=.45$, $\hat{\beta_0}=-15.28$
		
		\begin{figure}
			\includegraphics[width=5cm,height=3.5cm]{ExpectedObserved}
			\caption{$\hat{\mu}$ and $Y(\boldsymbol{s_i},t)$}
			\label{fig:ExpectedObserved}
		\end{figure}
		\item Approximations to marginal credible intervals, $\sigma^2 \in (2.31,2.94)$, $\theta_1 \in  (.135,.144)$, $\eta \in (.438,.458)$, and $\beta_0 \in (15.12,15.44)$
	\end{itemize}
\end{frame}


\begin{frame}
	\frametitle{Fit of RDSEM - Conditional on $\hat{\beta_0}=-15.5$}
	\begin{itemize}
		\item Posterior mode at $\hat{\alpha}=0.01$, $\hat{\beta}=.07$, $\hat{\sigma}^2=.44$, $\hat{\eta}=0.02$.
		
		\begin{figure}
			\includegraphics[width=5cm,height=3.5cm]{ObservedExpected2}
			\caption{$\hat{\mu}$ and $Y(\boldsymbol{s_i},t)$}
			\label{fig:Expect}
		\end{figure}
		\item Approximations to marginal credible intervals, $\alpha \in (.002,.017)$, $\beta \in (.05,.10)$, $\sigma^2 \in (.30,.57)$, $\eta \in (0,.05)$.
	\end{itemize}
\end{frame}



\section{Iraq Data}



\begin{frame}
	\frametitle{Geography - Neighborhood Structure}
	
	\begin{figure}[ht]
		\centering
		\includegraphics[width=7cm, height=5.5cm]{IncidentsOverPop}
		\label{fig:b}
	\end{figure}
	
	\begin{itemize}
		\item Districts corresponds to CIA distribution of tribes and recognized geographical borders
		\item 155 districts, maximum number of neighbors is 12
		\item Districts with no neighbors removed from sample
	\end{itemize}
\end{frame}


\begin{frame}
	\frametitle{Impact of Self-Excitation Parameter}
	\begin{figure}[ht]
		\centering
		\includegraphics[width=\textwidth]{Contageon}
		\caption{SCSEM - Single Node}
		\label{fig:SCSEMExample}
	\end{figure}
\end{frame}


\begin{frame}
	\frametitle{Sensitivity of $\beta$ parameter}
	\begin{figure}[h]
		\vspace*{.51cm}
		\centering
		\includegraphics[width=8cm]{betasensitivity}
		\label{fig:beta}
	\end{figure}
\end{frame}

\begin{frame}
	\frametitle{Impact of Self-Excitation Parameter}
	$\eta$ can be thought of as expected number of events an event causes in the next time period, $\eta \in (0,1)$
	\begin{figure}[ht]
		\centering
		\includegraphics[width=8cm]{Contageon3}
		\label{fig:SCSEMExample}
	\end{figure}
	
\end{frame}


\begin{frame}
	\frametitle{asdf}
	\begin{itemize}
		\item Data Model (Aerial Data, Discrete Time)
		\begin{align}
		& Y(\boldsymbol{s_i},t) | X(\boldsymbol{s_i},t), \eta  \sim \mathcal{P}\\
		& s_i =\text{Region in } \mathcal{R}^2 \\
		& t \in (1,2,3,...) \equiv \mathcal{T}
		\end{align}
		\item Process Model
		\begin{equation}
		X(\boldsymbol{s_i},t) | \theta \sim \mathcal{P}
		\end{equation}
		\item Parameter Model
		\begin{align}
		\theta \sim \mathcal{P}\\
		\eta \sim \mathcal{P}
		\end{align}
	\end{itemize}
\end{frame}


\begin{frame}
	\frametitle{Self-Exciting Models}
	\begin{itemize}
		\item Counting process originally introduced by Alan Hawkes
		\begin{align}
		& \text{Pr }\left(\Delta N(t)=1|N(s), s\leq t \right)=\Lambda(t)\Delta_t\\
		& \Lambda(t)= \nu + \int_{0}^{t} \mu (t-u) dN(u)
		\end{align}
		\item Positive correlation between intensity and counts
		\item Parent-Child Process
		\item Discretized form:
		\begin{align}
		& Y(t_i)\sim Pois(\lambda_i)\\
		& \lambda_i = \nu + \sum_{j < i} \kappa^{i-j}y(t_j)
		\end{align}
		
		\item Hawkes provided little guidance on $\nu$ (Background Intensity)
		
	\end{itemize}
\end{frame}



\begin{frame}
	\frametitle{Research Goals - Chapter 1}
	\textbf{ENDSTATE:} Framework for modeling the spatio-temporal diffusion of terrorism allowing for the possibility of self-excitation\newline
	\textbf{HOW:}
	\begin{itemize}
		\item Formulate models that reflect competing theories of spatio-temporal diffusion of violence
		\item Efficiently estimate parameters using Laplace Approximations 
		\item Conduct model assessment 
	\end{itemize}
	\textbf{WHY:} Successful modeling framework can be exported to future conflicts, allowing practitioners to better give policy recommendations to counter violent actors
\end{frame}

\begin{frame}
	\frametitle{Primary Contributions}
	\begin{itemize}
		\item Adapted Laplace approximation techniques of Integrated Nested Laplace Approximation (INLA) to conduct Approximate Bayesian inference for Self-Exciting models
		\item Analyze models that combine spatio-temporal diffusion with self-excitation
		\item Used theory on terrorism/crime diffusion to create and analyze real world terrorism data demonstrating how differing theories of diffusion lead to differing conclusions on self-excitation
	\end{itemize}
\end{frame}
%------------------------------------------------




\section{Modeling}


\begin{frame}
	\frametitle{Theory Driven Model of Spatio-Temporal Diffusion}
	\begin{itemize}
		\item Relationship between the spatio-temporal observations is of interest as opposed to exogenous covariates
		\begin{itemize}
			\item Relocation vs Escalation (Schutte \& Weidmann, 2011)
			\item Containment, Relocation, Escalation, and Flashpoints (Baudains, 2015)
		\end{itemize}
		\item Different forms of spatio-temporal evolution lead to different solutions
		\begin{itemize}
			\item Identify Root Causes and address (Political/Economic)
			\item Combat malicious actors from inside locale
			\item Isolate Population from outside influence
		\end{itemize}
	\end{itemize}
\end{frame}

\begin{frame}
	\frametitle{Models - Supporting Assumptions/Hypothesis}
	\begin{itemize}
		\item Spatial Only
		\begin{itemize}
			\item Devote more resources to identification of root causes
		\end{itemize}
		\item Spatially Correlated/Self-Exciting
		\begin{itemize}
			\item Identify and address local actors 
		\end{itemize}
		\item Reaction-Diffusion
		\begin{itemize}
			\item Isolate population
		\end{itemize}
		\item Reaction-Diffusion/Self-Exciting
		\begin{itemize}
			\item Isolate population and identify/combat malicious actors
		\end{itemize}
	\end{itemize}
\end{frame}


\begin{frame}
	\frametitle{Spatially Correlated Model}
	\begin{itemize}
		\item \textbf{Theory:} Exist common cause between geographically proximate locations
		\pause
		\item Let $Y(\boldsymbol{s_i},t)$ be number of recorded violent events at location $s_i$ at time $t$.
		\item $t\in\{1,2,...,n\}$, $i\in\{1,2,...,s\}$
		\item Let latent state $X(\boldsymbol{s_i},t)$ be continuous measure of violence tendency, function of neighboring levels of violence
	\end{itemize}
	
	\begin{align}
	& Y(\boldsymbol{s_i},t)|\mu(\boldsymbol{s_i},t) \sim \text{Pois }(\mu(\boldsymbol{s_i},t)) \label{eq:Full Model1}\\
	& \mu(\boldsymbol{s_i},t) = \exp(X(\boldsymbol{s_i},t)) \nonumber \\
	& X(\boldsymbol{s_i},t) = \theta_1 \sum_{\boldsymbol{s_j}\in N(\boldsymbol{s_i})}X(\boldsymbol{s_j},t) + \epsilon(\boldsymbol{s_i},t) \nonumber\\
	&\epsilon(\boldsymbol{s_i},t) \sim Gau(0,\sigma^2) \nonumber
	\end{align}
\end{frame}


\begin{frame}
	\frametitle{Model Properties}
	\begin{itemize}
		\item Let $\boldsymbol{H}$ denote the spatial neighborhood matrix such that $H_{i,j}=H_{j,i}=1$ if $\boldsymbol{s_i}$ and $\boldsymbol{s_j}$ are neighbors
		\item The joint distribution of $\boldsymbol{X}\sim Gau (\boldsymbol{0},(\boldsymbol{I}_{ns,ns}-\theta_1\boldsymbol{I}_{n,n} \otimes \boldsymbol{H})^{-1}\boldsymbol{L}(\boldsymbol{I}_{ns,ns}-\theta_1\boldsymbol{I}_{n,n} \otimes \boldsymbol{H})^{-1})$ where $\boldsymbol{L}=\text{diag}(\sigma^2,...,\sigma^2)$
		\pause
		\item Modeling limitation, $\theta_1 \in (\psi_{(1)}^{-1},\psi_{(n)}^{-1})$ where $\psi$ are the eigenvalues of $\boldsymbol{H}$
		\pause
		\item No temporal correlation
		\item Propensity of a location to experience violence is spatially correlated 
		\item To combat - focus on identification of root causes inherent in geography
	\end{itemize}
\end{frame}

\begin{frame}
	\frametitle{Spatially Correlated Self-Exciting Model}
	\begin{itemize}
		\item \textbf{Theory:} Exist common cause between geographically similar locations, regions that experience uptick in violence likely to have short term self-excitation
		\pause
		\item Mixture of two processes that influence expectation : LGCP and Hawkes process
		\item Hawkes process letting $\mu(t-u)=\eta$ if $(t-u)=1$, $0$ otherwise
		\begin{align}
		& Y(\boldsymbol{s_i},t)|\mu(\boldsymbol{s_i},t) \sim \text{Pois }(\mu(\boldsymbol{s_i},t)) \label{eq:Full Model}\\
		& \mu(\boldsymbol{s_i},t) = \exp(X(\boldsymbol{s_i},t)) + \highlight{\eta Y(\boldsymbol{s_i},t-1)} \nonumber \\
		& X(\boldsymbol{s_i},t) = \theta_1 \sum_{\boldsymbol{s_j}\in N(\boldsymbol{s_i})}X(\boldsymbol{s_j},t) + \epsilon(\boldsymbol{s_i},t) \nonumber\\
		&\epsilon(\boldsymbol{s_i},t) \sim Gau(0,\sigma^2) \nonumber
		\end{align}
		
		
	\end{itemize}
\end{frame}



\begin{frame}
	\frametitle{Reaction Diffusion Self-Exciting Model}
	\begin{itemize}
		\item \textbf{Theory:} Spatio-temporal diffusion is due to physical movement of actors
		\item $X(\boldsymbol{s_i},t)$ is continuous measure of violence due to unobserved actors at location $(\boldsymbol{s_i},t)$
		\begin{equation}
		\frac{\partial X(\boldsymbol{s_i},t)}{\partial t}=\frac{\beta}{|N(s_i)|} \triangle X(\boldsymbol{s_i},t)-\alpha X(\boldsymbol{s_i},t) \label{eq:Reaction}
		\end{equation}
		\pause
		\item To discretize: let $\Gamma$ be graphical Laplacian. 
		\[
		\Gamma(s_i,s_j)
		\begin{cases} 
		\hfill -|N(s_i)|   \hfill &  j=i \\
		\hfill 1 \hfill & j\in N(s_i)  \\
		\hfill 0 \hfill& \text{Otherwise}
		\end{cases}
		\]
	\end{itemize}
	\begin{equation}
	\boldsymbol{X}_{t}-\boldsymbol{X}_{t-1}=\beta \text{ diag}\left( \frac{1}{|N_{s_i}|}\right) \Gamma \boldsymbol{X}_{t-1}-\alpha \boldsymbol{X}_{t-1}
	\end{equation}
	
\end{frame}	



\begin{frame}{Differences in Space/Time Diffusion of $X(\boldsymbol{s_i},t)$}
	\begin{center}
		SCSEM - Conditional on Node 1 = 10
	\end{center}
	\begin{center}
		\begin{tikzpicture}[>=latex']
		\tikzset{block/.style= { rectangle, align=left,minimum width=.2cm,minimum height=.1cm},
			rblock/.style={draw, shape=rectangle,rounded corners=1.5em,align=center,minimum width=.2cm,minimum height=.1cm},
			input/.style={ % requires library shapes.geometric
				draw,
				trapezium,
				trapezium left angle=60,
				trapezium right angle=120,
				minimum width=2cm,
				align=center,
				minimum height=1cm
			},
		}
		\node [block, fill=orange!90]  (x1) {\footnotesize 10};
		\node [block, right = .1cm of x1, fill=blue!50] (x2) {\footnotesize 5};
		\node [block, right = .1cm of x2,fill=blue!20] (x3) {\footnotesize 2};
		\node [block, right = .1cm of x3, fill=blue!10] (x4) {\footnotesize 1};
		\node [block, below = .1cm of x1, fill=blue!50] (y1) {\footnotesize 5};
		\node [block, below = .1cm of x2, fill=blue!50] (y2) {\footnotesize 4};
		\node [block, below = .1cm of x3, fill=blue!30] (y3) {\footnotesize 3};
		\node [block, below = .1cm of x4,fill=blue!20] (y4) {\footnotesize 2};
		\node [block, below = .1cm of y1,fill=blue!20] (z1) {\footnotesize 2};
		\node [block, below = .1cm of y2, fill=blue!30] (z2) {\footnotesize 3};
		\node [block, below = .1cm of y3, fill=blue!30] (z3) {\footnotesize 2};
		\node [block, below = .1cm of y4, fill=blue!10] (z4) {\footnotesize 1};
		\node [block, below = .1cm of z1, fill=blue!10] (q1) {\footnotesize 1};
		\node [block, below = .1cm of z2, fill=blue!10] (q2) {\footnotesize 1};
		\node [block, below = .1cm of z3, fill=blue!10] (q3) {\footnotesize 1};
		\node [block, below = .1cm of z4, fill=blue!10] (q4) {\footnotesize 1};
		\node [block, above = .1cm of x2] {\footnotesize Time 1};
		\end{tikzpicture}
		\quad
		\begin{tikzpicture}[>=latex']
		\tikzset{block/.style= { rectangle, align=left,minimum width=.2cm,minimum height=.1cm},
			rblock/.style={draw, shape=rectangle,rounded corners=1.5em,align=center,minimum width=.2cm,minimum height=.1cm},
			input/.style={ % requires library shapes.geometric
				draw,
				trapezium,
				trapezium left angle=60,
				trapezium right angle=120,
				minimum width=2cm,
				align=center,
				minimum height=1cm
			},
		}
		\node [block]  (x1) {\footnotesize 0};
		\node [block, right = .1cm of x1] (x2) {\footnotesize 0};
		\node [block, right = .1cm of x2] (x3) {\footnotesize 0};
		\node [block, right = .1cm of x3] (x4) {\footnotesize 0};
		\node [block, below = .1cm of x1] (y1) {\footnotesize 0};
		\node [block, below = .1cm of x2] (y2) {\footnotesize 0};
		\node [block, below = .1cm of x3] (y3) {\footnotesize 0};
		\node [block, below = .1cm of x4] (y4) {\footnotesize 0};
		\node [block, below = .1cm of y1] (z1) {\footnotesize 0};
		\node [block, below = .1cm of y2] (z2) {\footnotesize 0};
		\node [block, below = .1cm of y3] (z3) {\footnotesize 0};
		\node [block, below = .1cm of y4] (z4) {\footnotesize 0};
		\node [block, below = .1cm of z1] (q1) {\footnotesize 0};
		\node [block, below = .1cm of z2] (q2) {\footnotesize 0};
		\node [block, below = .1cm of z3] (q3) {\footnotesize 0};
		\node [block, below = .1cm of z4] (q4) {\footnotesize 0};
		\node [block, above = .1cm of x2] {\footnotesize Time 2};
		\end{tikzpicture}
	\end{center}
	\begin{center}
		RDSEM - Conditional on Node 1 = 10
	\end{center}
	\quad
	\begin{center}
		\begin{tikzpicture}[>=latex']
		\tikzset{block/.style= { rectangle, align=left,minimum width=.2cm,minimum height=.1cm},
			rblock/.style={draw, shape=rectangle,rounded corners=1.5em,align=center,minimum width=.2cm,minimum height=.1cm},
			input/.style={ % requires library shapes.geometric
				draw,
				trapezium,
				trapezium left angle=60,
				trapezium right angle=120,
				minimum width=2cm,
				align=center,
				minimum height=1cm
			},
		}
		\node [block, fill=orange!90]  (x1) {\footnotesize 10};
		\node [block, right = .1cm of x1,fill=blue!10] (x2) {\footnotesize 1};
		\node [block, right = .1cm of x2, fill=blue!10] (x3) {\footnotesize 1};
		\node [block, right = .1cm of x3] (x4) {\footnotesize 0};
		\node [block, below = .1cm of x1,fill=blue!10] (y1) {\footnotesize 1};
		\node [block, below = .1cm of x2, fill=blue!10] (y2) {\footnotesize 1};
		\node [block, below = .1cm of x3] (y3) {\footnotesize 0};
		\node [block, below = .1cm of x4] (y4) {\footnotesize 0};
		\node [block, below = .1cm of y1, fill=blue!10] (z1) {\footnotesize 1};
		\node [block, below = .1cm of y2] (z2) {\footnotesize 0};
		\node [block, below = .1cm of y3] (z3) {\footnotesize 0};
		\node [block, below = .1cm of y4] (z4) {\footnotesize 0};
		\node [block, below = .1cm of z1] (q1) {\footnotesize 0};
		\node [block, below = .1cm of z2] (q2) {\footnotesize 0};
		\node [block, below = .1cm of z3] (q3) {\footnotesize 0};
		\node [block, below = .1cm of z4] (q4) {\footnotesize 0};
		\node [block, above = .1cm of x2] {\footnotesize Time 1};
		\end{tikzpicture}
		\qquad
		\begin{tikzpicture}[>=latex']
		\tikzset{block/.style= { rectangle, align=left,minimum width=.2cm,minimum height=.1cm},
			rblock/.style={draw, shape=rectangle,rounded corners=1.5em,align=center,minimum width=.2cm,minimum height=.1cm},
			input/.style={ % requires library shapes.geometric
				draw,
				trapezium,
				trapezium left angle=60,
				trapezium right angle=120,
				minimum width=2cm,
				align=center,
				minimum height=1cm
			},
		}
		\node [block,fill=blue!30]  (x1) {\footnotesize 3};
		\node [block, right = .1cm of x1, fill=blue!30] (x2) {\footnotesize 3};
		\node [block, right = .1cm of x2] (x3) {\footnotesize 0};
		\node [block, right = .1cm of x3] (x4) {\footnotesize 0};
		\node [block, below = .1cm of x1, fill=blue!30] (y1) {\footnotesize 3};
		\node [block, below = .1cm of x2, fill=blue!10] (y2) {\footnotesize 1};
		\node [block, below = .1cm of x3] (y3) {\footnotesize 0};
		\node [block, below = .1cm of x4] (y4) {\footnotesize 0};
		\node [block, below = .1cm of y1] (z1) {\footnotesize 0};
		\node [block, below = .1cm of y2] (z2) {\footnotesize 0};
		\node [block, below = .1cm of y3] (z3) {\footnotesize 0};
		\node [block, below = .1cm of y4] (z4) {\footnotesize 0};
		\node [block, below = .1cm of z1] (q1) {\footnotesize 0};
		\node [block, below = .1cm of z2] (q2) {\footnotesize 0};
		\node [block, below = .1cm of z3] (q3) {\footnotesize 0};
		\node [block, below = .1cm of z4] (q4) {\footnotesize 0};
		\node [block, above = .1cm of x2] {\footnotesize Time 2};
		\end{tikzpicture}
		\qquad
		\begin{tikzpicture}[>=latex']
		\tikzset{block/.style= { rectangle, align=center,minimum width=.2cm,minimum height=.1cm},
			rblock/.style={draw, shape=rectangle,rounded corners=1.5em,align=center,minimum width=.2cm,minimum height=.1cm},
			input/.style={ % requires library shapes.geometric
				draw,
				trapezium,
				trapezium left angle=60,
				trapezium right angle=120,
				minimum width=2cm,
				align=center,
				minimum height=1cm
			},
		}
		
		\node [block, fill=blue!20]  (x1) {\footnotesize 2};
		\node [block, right = .1cm of x1, fill=blue!10] (x2) {\footnotesize 1};
		\node [block, right = .1cm of x2, fill=blue!10] (x3) {\footnotesize 1};
		\node [block, right = .1cm of x3] (x4) {\footnotesize 0};
		\node [block, below = .1cm of x1, fill=blue!10] (y1) {\footnotesize 1};
		\node [block, below = .1cm of x2, fill=blue!10] (y2) {\footnotesize 1};
		\node [block, below = .1cm of x3] (y3) {\footnotesize 0};
		\node [block, below = .1cm of x4] (y4) {\footnotesize 0};
		\node [block, below = .1cm of y1, fill=blue!10] (z1) {\footnotesize 1};
		\node [block, below = .1cm of y2] (z2) {\footnotesize 0};
		\node [block, below = .1cm of y3] (z3) {\footnotesize 0};
		\node [block, below = .1cm of y4] (z4) {\footnotesize 0};
		\node [block, below = .1cm of z1] (q1) {\footnotesize 0};
		\node [block, below = .1cm of z2] (q2) {\footnotesize 0};
		\node [block, below = .1cm of z3] (q3) {\footnotesize 0};
		\node [block, below = .1cm of z4] (q4) {\footnotesize 0};
		\node [block, above = .1cm of x2] {\footnotesize Time 3};
		\end{tikzpicture}\\
	\end{center}
\end{frame}


\begin{frame}
	\frametitle{RDSEM}
	\begin{itemize}
		\item $Y(\boldsymbol{s_i},t)$ remains count of events at location $(\boldsymbol{s_i},t)$
		\item Addition of random noise:
	\end{itemize}
	\begin{align}
	& Y(\boldsymbol{s_i},t) \sim \text{Pois }(\mu(\boldsymbol{s_i},t)) \label{eq:ReacDiffuse Model}\\
	& \mu(\boldsymbol{s_i},t) = \exp(X(\boldsymbol{s_i},t)) + \eta Y(\boldsymbol{s_i},t-1) \nonumber \\
	& \small X(\boldsymbol{s_i},t) = \frac{\highlight{\beta}}{|N(s_i)|} \sum_{\boldsymbol{s_j}\in N(\boldsymbol{s_i})}X(\boldsymbol{s_j},\highlighter{t-1}) + (1-\highlight{\beta}-\alpha)  X(\highlighter{\boldsymbol{s_i},t-1}) + \epsilon(\boldsymbol{s},t) \nonumber\\
	&\epsilon(\boldsymbol{s},t) \sim Gau(0,\sigma^2) \nonumber
	\end{align}
	\pause
	\begin{itemize}
		\item Let $\boldsymbol{M}=\beta \text{ diag}\left( \frac{1}{|N_{s_i}|}\right) \Gamma + (1-\alpha) \boldsymbol{I}_{s,s}$
		\item $\boldsymbol{X}_t = \boldsymbol{M} \boldsymbol{X}_{t-1} + \boldsymbol{\epsilon}$, $\boldsymbol{\epsilon} \sim Gau(\boldsymbol{0},\sigma^2 \boldsymbol{I})$
		\item VAR(1) model requires maximum eigenvalue of $\boldsymbol{M}$ less than 1
		
	\end{itemize}
	
\end{frame}
\begin{frame}
	\frametitle{Parameter Space of RDSEM}
	$\boldsymbol{M}=\beta \text{ diag}\left( \frac{1}{|N_{s_i}|}\right) \Gamma + (1-\alpha) \boldsymbol{I}_{s,s}$
	\begin{itemize}
		\item Maximum eigenvalue of $\beta\text{ diag}\left( \frac{1}{|N_{s_i}|}\right) \Gamma$ is zero
		\item Minimum eigenvalue is $-2\beta$
		\begin{itemize}
			\item Due to properties of normalized Laplacian matricies
		\end{itemize}
		\item Implies maximum eigenvalue of $\boldsymbol{M}$ is $(1-\alpha)$ and minimum is  $-2\beta+(1-\alpha)$
		\item $\alpha \in (0,1)$, $\beta \in (\frac{-\alpha}{2},\frac{2-\alpha}{2})$ is allowable parameter space to ensure $\boldsymbol{M}$ is well-behaved
	\end{itemize}
\end{frame}

\begin{frame}
	\frametitle{Latent State Properties}
	\begin{itemize}
		\item Model is also VAR(1) with explicit joint distribution
		\item $\Sigma_s$ is spatial covariance - time invariant
		\item Let $\boldsymbol{M}$ as above,then $\boldsymbol{X}$ has explicit solution $X\sim Gau(\boldsymbol{0},\boldsymbol{\Sigma})$, letting $\text{vec}(\Sigma_s)=\left(\boldsymbol{I}_{s^2,s^2}-\boldsymbol{M}\otimes\boldsymbol{M}\right)^{-1}\text{vec}\left(\sigma^2 \boldsymbol{I}_{s,s}\right)$,
		\begin{equation}
		\Sigma=
		\left[
		\begin{array}{c|c|c|c}
		\Sigma_s & M \Sigma_s & ... & M^n\Sigma_s \\
		\hline
		M \Sigma_s & \Sigma_s & ... & M^{n-1}\Sigma_s\\
		\hline
		... & ... & ... & ...\\
		\hline
		M^n\Sigma_s & M^{n-1}\Sigma_s & ... & \Sigma_s
		\end{array} 
		\right]\label{eq:Sig}
		\end{equation}
		\pause
		\item This is not especially helpful due to dimension of $\left(\boldsymbol{I}_{s^2,s^2}-\boldsymbol{M}\otimes\boldsymbol{M}\right)$
		\item Even storing/computing is problematic $\dim (\Sigma_s)=s \times s$, $\dim (\Sigma)= ns \times ns$, Iraq data $n=96$, $s=155$
	\end{itemize}
	
\end{frame}	
\begin{frame}
	\frametitle{Latent State Properties}
	Precision Matrix has explicit solution as well due to VAR(1) structure
	\begin{equation}
	\Sigma^{-1}=
	\left[
	\begin{array}{c|c|c|c|c}
	\boldsymbol{I}_{n,n} & -M & \boldsymbol{0}&  ... & ... \\
	\hline
	- M^T  & M^T M +\boldsymbol{I}_{n,n} & - M & \boldsymbol{0} & ... \\
	\hline
	\boldsymbol{0} &- M^T  & M^T M +\boldsymbol{I}_{n,n} & - M & ...\\
	\hline
	...&...&...&...&...\\
	\hline
	\boldsymbol{0} & ... & - M^T  & M^T M +\boldsymbol{I}_{n,n} & - M\\
	\hline
	\boldsymbol{0} & ... & ... & -M^T & \boldsymbol{I}_{n,n}
	\end{array} 
	\right] \frac{1}{\sigma^2}\label{eq:Prec}
	\end{equation}
	\begin{itemize}
		\item Sparse precision matrix
		\item Process model contains both spatial and temporal diffusion
		\item Countering strategy suggests isolation/address multiple regions simultaneously
	\end{itemize}
	
	
\end{frame}	


\section{Computation}

\begin{frame}
	\frametitle{Traditional INLA}
	\begin{itemize}
		\item 2009 work by Rue et al.  Implemented in R package 'R-INLA'
		\item Assumes latent Gaussian structure with Gaussian $\beta_0$, $\boldsymbol{\beta}$, $X(\boldsymbol{s_i},t)$
		\begin{align*}
		\mu(\boldsymbol{s_i,t})&=\exp(\lambda(\boldsymbol{s_i},t))\\
		\lambda(\boldsymbol{s_i},t) &= \beta_0 + \boldsymbol{Z}^t \boldsymbol{\beta} + X(\boldsymbol{s_i,t})\\
		X(\boldsymbol{s_i,t})& \sim Gau(\boldsymbol{0},\Sigma(\theta))\\
		\end{align*}
		\item Let $\boldsymbol{\zeta}=(\beta_0,\boldsymbol{\beta},\boldsymbol{X})^T$:
		\begin{align*}
		&\tilde{\pi}(\boldsymbol{\theta}|\boldsymbol{Y})\propto \frac{\pi(\boldsymbol{\zeta},\boldsymbol{\theta},\boldsymbol{y})}{\tilde{\pi}_G (\boldsymbol{\zeta}|\boldsymbol{\theta},\boldsymbol{Y})}\bigg\vert_{\zeta=\zeta^*(\boldsymbol{\theta})}
		\\
		&\tilde{\pi}(\beta_0|\boldsymbol{\theta},\boldsymbol{Y})\propto \frac{\pi(\boldsymbol{\zeta},\boldsymbol{\theta},\boldsymbol{y})}{\tilde{\pi}_G (\boldsymbol{\beta},\boldsymbol{X}|\beta_0,\boldsymbol{\theta},\boldsymbol{Y})}\bigg|_{(\boldsymbol{\beta},\boldsymbol{X})=(\boldsymbol{\beta},\boldsymbol{X})^*(\beta_0,\boldsymbol{\theta})}
		\\
		&\tilde{\pi}(\beta_0|\boldsymbol{Y}) = \sum \tilde{\pi}(\boldsymbol{\theta}|\boldsymbol{Y})\tilde{\pi}(\beta_0|\boldsymbol{\theta},\boldsymbol{Y})
		\end{align*}
	\end{itemize}
\end{frame}


\begin{frame}
	\frametitle{Issues with Traditional INLA for Self-Exciting Models}
	Recall:
	\begin{align*}
	\mu(\boldsymbol{s_i},t) & = \exp(X(\boldsymbol{s_i},t)) + \eta Y(\boldsymbol{s_i},t-1)\\
	& \eta \in (0,1)
	\end{align*}
	\begin{itemize}
		\item No clear method for inference of non-Gaussian $\eta$ term not linearly related to $X(\boldsymbol{s_i},t)$
		\item $\eta$ support does not allow Gaussian prior
		\item No Built in methods for dealing with non-separable space-time covariance structure
		\item \textbf{Fix}: We will treat $\eta$ as a process model parameter (Diffusion Parameters) and use single Laplace Approximation
	\end{itemize}
\end{frame}

\begin{frame}
	\frametitle{Approximate Bayesian Computation Using Laplace Approximation}
	\begin{itemize}
		\item Use Laplace Approximation to approximate $\pi(\boldsymbol{\theta}|y)$
		\begin{align*}
		\pi(\boldsymbol{\theta}|\boldsymbol{y})& =\frac{\pi(\boldsymbol{x},\boldsymbol{\theta}|\boldsymbol{y})}{\pi(\boldsymbol{x}|\boldsymbol{\theta},\boldsymbol{y})}\\
		& = \frac{\pi(\boldsymbol{y}|\boldsymbol{x},\boldsymbol{\theta})\pi(\boldsymbol{x},\boldsymbol{\theta})}{\pi(\boldsymbol{y})}\frac{1}{\pi(\boldsymbol{x}|\boldsymbol{\theta},\boldsymbol{y})}\\
		& \propto \frac{\pi(\boldsymbol{y}|\boldsymbol{x},\boldsymbol{\theta})\pi(\boldsymbol{x}|\boldsymbol{\theta})\pi(\boldsymbol{\theta})}{\pi(\boldsymbol{x}|\boldsymbol{\theta},\boldsymbol{y})}
		\end{align*}
		\item Evaluate by taking Gaussian approximation to denominator 
		\begin{equation}
		\tilde{\pi}(\boldsymbol{\theta}|\boldsymbol{y})\propto \frac{\pi(\boldsymbol{x},\boldsymbol{\theta},\boldsymbol{y})}{\tilde{\pi}_G (\boldsymbol{x}|\boldsymbol{\theta},\boldsymbol{y})}|_{x=x^*(\boldsymbol{\theta})} \label{eq:LaplaceINLA}
		\end{equation}
		\item Moreover, holds for any $\boldsymbol{x}$, so chose convenient one
		\item Related to Tierney and Kadane Laplace Approximation to posterior marginal
	\end{itemize}
\end{frame}

\begin{frame}
	\frametitle{Full Conditional for Latent Variables}
	\begin{itemize}
		\item In both the Spatially Correlated Self-Exciting Model (SCSEM) and the Reaction Diffusion Self-Exciting model (RDSEM), the full conditional for the latent state is
		\begin{equation}
		\footnotesize\pi(\boldsymbol{X}|\boldsymbol{Y},\boldsymbol{\theta}) \propto \exp\left(-\frac{1}{2}\boldsymbol{X}^T \boldsymbol{Q(\boldsymbol{\theta})}\boldsymbol{X} + \sum_{s_i,t} \log \pi\left(Y(\boldsymbol{s_i},t)|X(\boldsymbol{s_i},t),\eta,Y(\boldsymbol{s_i},t-1)\right)\right)\label{eq:FullCond}
		\end{equation} 
		\item Where $\boldsymbol{\theta}=(\theta_1,\sigma^2,\eta)^T$ for SCSEM and $\boldsymbol{\theta}=(\alpha,\beta,\sigma^2,\eta)^T$ for RDSEM
	\end{itemize}
\end{frame}

\begin{frame}
	\frametitle{Laplace Approximations for Spatio-Temporal Self-Exciting Models}
	\begin{itemize} 
		\item Expand $\pi\left(Y(\boldsymbol{s_i},t)|X(\boldsymbol{s_i},t),\eta,Y(\boldsymbol{s_i},t-1)\right)$ as a function of $X$ about a guess for the mode, $\boldsymbol{\mu_0}$
		\item Collect like terms
		\begin{equation}
		\footnotesize\pi(\boldsymbol{X}|\boldsymbol{Y},\boldsymbol{\theta}) \propto \exp\left(-\frac{1}{2}\boldsymbol{X}^T\left( \boldsymbol{Q^*(\boldsymbol{\theta})|\mu_0}\right)\boldsymbol{X} + \boldsymbol{X}^T\left( \boldsymbol{B^*}(\boldsymbol{\theta})|\boldsymbol{\mu_0}\right)\right) \label{eq:FullCondExpand}
		\end{equation}
		\item Impact of $\eta$ seen in
		\begin{equation}
		\boldsymbol{Q^*(\boldsymbol{\theta})|\mu_0}=\boldsymbol{Q(\boldsymbol{\theta})}+\text{diag }\left(-\frac{\partial^2\log \pi\left(Y(\boldsymbol{s_i},t)\right)}{\partial X(\boldsymbol{s_i},t)^2}\right)\Bigr|_{X(\boldsymbol{s_i},t)=\mu_0(\boldsymbol{s_i},t)} \label{eq:Precision at Mode}
		\end{equation}
		\item Find mode by iteratively solving $\left(Q^*(\boldsymbol{\theta})|\boldsymbol{\mu_n}\right)\boldsymbol{\mu_{n+1}}=\boldsymbol{B^*}(\boldsymbol{\theta}|\boldsymbol{\mu_n})$
	\end{itemize}
	
\end{frame}


\begin{frame}
	\frametitle{Exploring $\pi(\boldsymbol{\theta}|\boldsymbol{Y})$}
	\begin{itemize}
		\item Need posterior density of $\theta|\boldsymbol{Y}$
		
		\begin{equation}
		\tilde{\pi}(\boldsymbol{\theta}|\boldsymbol{y})\propto \frac{\pi(\boldsymbol{x},\boldsymbol{\theta},\boldsymbol{y})}{\tilde{\pi}_G (\boldsymbol{x}|\boldsymbol{\theta},\boldsymbol{y})}|_{x=x^*(\boldsymbol{\theta})}
		\end{equation}
		\begin{itemize}
			\item Find posterior mode of $\log\tilde{\pi}(\boldsymbol{\theta}|\boldsymbol{y})$ 
			\item Numerically approximate Hessian at posterior mode
			\item Eigen decomposition of negative inverse of the Hessian, $\boldsymbol{V}\boldsymbol{\Lambda}\boldsymbol{V}^T$
			\item Calculate $\boldsymbol{\theta}^{new}=\boldsymbol{\theta}^*+\boldsymbol{V}\boldsymbol{\Lambda}^{1/2}\boldsymbol{z}$
			\item Compare $\log \pi(\boldsymbol{\theta}^{new}|\boldsymbol{Y})$ to $\log \pi(\boldsymbol{\theta}^*|\boldsymbol{Y})$, if it is sufficiently small, stop, or else increase $\boldsymbol{z}$
		\end{itemize}
		\item Set of $\boldsymbol{\theta}^{new}|\boldsymbol{Y}$ approximates the posterior for the diffusion parameters
		
	\end{itemize}
	
	
\end{frame}

\section{Simulation}



\begin{frame}
	\frametitle{Simulation}
	\begin{itemize}
		\item 4x4 Grid Rook Neighbors - 100 time points
		\item Gradient descent with step-halving approx. 5-10 min to find posterior mode on Surface Pro 4 i5
		\item 230 points from posterior found in additional 3-5 min
		\item 95\% credible intervals
		\begin{itemize}
			\item SCSEM $\theta_1=.25,\sigma^2=1.1,\eta=.3$
			\begin{itemize}
				\item $\theta_1\in(.14,.27)$
				\item $\sigma^2\in(.88,1.22)$
				\item $\eta \in(.28,.33)$
			\end{itemize}
			\item For RDSEM, $\alpha=.1$, $\beta=.23$, $\sigma^2=1.1$, $\eta=.4$
			\begin{itemize}
				\item $\alpha \in (.04,.19)$
				\item $\beta \in (.20,.25)$
				\item $\sigma^2 \in (.98,1.05)$
				\item $\eta \in (.37,.42)$
			\end{itemize}
		\end{itemize}
	\end{itemize}
\end{frame}



\begin{frame}
	\frametitle{Statistical Models - SCSEM}
	
	\begin{align}
	& Y(\boldsymbol{s_i},t)|\mu(\boldsymbol{s_i},t) \sim \text{Pois }(\mu(\boldsymbol{s_i},t)) \label{eq:IZ Model}\\
	& \mu(\boldsymbol{s_i},t) = \exp(\beta_0+\log \text{Pop}(\boldsymbol{s_i,t})+X(\boldsymbol{s_i},t)) + \eta Y(\boldsymbol{s_i},t-1) \nonumber \\
	& X(\boldsymbol{s_i},t) = \theta_1 \sum_{\boldsymbol{s_j}\in N(\boldsymbol{s_i})}X(\boldsymbol{s_j},t) + \epsilon(\boldsymbol{s_i},t) \nonumber\\
	&\epsilon(\boldsymbol{s_i},t) \sim Gau(0,\sigma^2) \nonumber
	\end{align}
	\begin{itemize}
		\item Aggregate Data Over Region and Month
		\item 96 Months, 155 Districts
		\item $Y(\boldsymbol{s_i},t)$ is number of incidents in district $s_i$ at time $t$
		\item Impact of $\beta_0$ and off-set
	\end{itemize}
\end{frame}


\begin{frame}
	\frametitle{SCSEM}
	$\pi(\boldsymbol{X},\boldsymbol{\theta},\boldsymbol{Y})= \pi(\boldsymbol{Y}|\beta_0,\boldsymbol{X},\eta,\boldsymbol{\theta})\pi(\beta_0)\pi(\eta)\pi(\boldsymbol{X}|\theta_1,\sigma^2)\pi(\theta_1)\pi(\sigma^2)$ 
	
	\textbf{Priors}
	\begin{itemize}
		\item $\pi(\sigma) \sim \text{Half Cauchy}(25)$
		\item  $\pi(\theta_1)\sim\text{Uniform }(\psi_{(1)}^{-1},\psi_{(n)}^{-1})$ where $\psi_{(i)}$ is the $i$th largest eigenvalue of $\boldsymbol{H}$
		\item $\pi(\beta_0)\sim\text{Normal }(0,10000)$
		\item Sensitivity Analysis performed on precision of Half Cauchy and variance of $\beta_0$
	\end{itemize}
	\pause
	\textbf{Parameter Estimates}
	\begin{itemize}
		\item Posterior mode at $\hat{\theta}_1=.126$, $\hat{\sigma}^2=1.66$, $\hat{\eta}=.43$, $\hat{\beta_0}=-14.27$
		\item Approximations to marginal credible intervals,  $\theta_1 \in  (.118,.128)$,$\sigma^2 \in (1.39,1.92)$, $\eta \in (.38,.48)$, and $\beta_0 \in (-14.2,-14.3)$
	\end{itemize}
	
\end{frame}




\begin{frame}
	\frametitle{Statistical Models - RDSEM}
	
	\begin{itemize}
		\item Motivating PDE
	\end{itemize}
	
	\begin{equation}
	\frac{\partial X(\boldsymbol{s_i},t)}{\partial t}=\frac{1}{|N(s_i)|}\beta \triangle X(\boldsymbol{s_i},t)-\alpha X(\boldsymbol{s_i},t)
	\end{equation}
	
	
	\begin{align}
	& Y(\boldsymbol{s_i},t) \sim \text{Pois }(\mu(\boldsymbol{s_i},t))\\
	& \mu(\boldsymbol{s_i},t) = \exp(\beta_0+\log \text{Pop}(\boldsymbol{s_i,t})+X(\boldsymbol{s_i},t)) + \eta Y(\boldsymbol{s_i},t-1) \nonumber \\
	& \small X(\boldsymbol{s_i},t) = \theta_2 (s_i) \sum_{\boldsymbol{s_j}\in N(\boldsymbol{s_i})}X(\boldsymbol{s_j},t-1) + \theta_3 X(\boldsymbol{s_i},t-1) + \epsilon(\boldsymbol{s},t) \nonumber\\
	&\epsilon(\boldsymbol{s},t) \sim N(0,\sigma^2) \nonumber
	\end{align}
	\begin{itemize}
		\item Where $\theta_2(s_i)= \frac{1}{|N(s_i)|}\beta$, $\theta_3=1-\alpha-\beta$
	\end{itemize}
\end{frame}

\begin{frame}
	\frametitle{RDSEM}
	$\pi(\boldsymbol{X},\boldsymbol{\theta},\boldsymbol{Y})=\pi(\boldsymbol{Y}|\boldsymbol{X},\eta,\boldsymbol{\theta})\pi(\beta_0)\pi(\eta)\pi(\boldsymbol{X}|\alpha,\beta,\sigma^2)\pi(\beta|\alpha)\pi(\alpha)\pi(\sigma^2)$
	\textbf{Priors}
	\begin{itemize}
		\item $\pi(\sigma) \sim \text{Half Cauchy}(25)$
		\item $\pi(\alpha)\sim \text{Uniform }(0,1)$
		\item $\pi(\beta|\alpha)\sim \text{Uniform }(-\frac{\alpha}{2},1-\frac{\alpha}{2})$
		\item $\pi(\eta)\sim \text{Uniform }(0,1)$
	\end{itemize}
	\textbf{Parameter Estimates}
	\begin{itemize}
		\item Posterior mode at $\hat{\alpha}=0.01$, $\hat{\beta}=.07$, $\hat{\sigma}^2=.44$, $\hat{\eta}=0.02$.
		\item Approximations to marginal credible intervals, $\alpha \in (.002,.017)$, $\beta \in (.04,.10)$, $\sigma^2 \in (.30,.57)$, $\eta \in (0,.05)$.
		\begin{itemize}
			\item Note: Profiled for $\beta_0$ so intervals are too optimistic
		\end{itemize}
		
	\end{itemize}
	
\end{frame}


\begin{frame}
	\frametitle{Little Parameter Value - Big Impact}
	\begin{itemize}
		\item Deterministic three node system, $\alpha=0$, $\eta=0$, $\beta=.07$, $\beta_0 = -15.5$, $\log\text{ Pop}=11.5$, 
	\end{itemize}
	\begin{align*} 
	&\exp(-4+X(\boldsymbol{s_i},t))\\ & X(\boldsymbol{s_2},0)=6.3\\
	& X(\boldsymbol{s_1},0)=X(\boldsymbol{s_1},0)=5\\
	&\boldsymbol{X}_t - \boldsymbol{X}_{t-1}=-\frac{.07}{|N(s_i)|} \Gamma \boldsymbol{X}_{t-1}
	\end{align*}
	
	\begin{center}
	\end{center}
	\begin{center}
		\begin{tikzpicture}[>=latex']
		\tikzset{block/.style= { rectangle, align=left,minimum width=.2cm,minimum height=.1cm},
			rblock/.style={draw, shape=rectangle,rounded corners=1.5em,align=center,minimum width=.2cm,minimum height=.1cm},
			input/.style={ % requires library shapes.geometric
				draw,
				trapezium,
				trapezium left angle=60,
				trapezium right angle=120,
				minimum width=2cm,
				align=center,
				minimum height=1cm
			},
		}
		\node [block,draw]  (x1) {2.7};
		\node [block, right = .3cm of x1,draw] (x2) {10};
		\node [block, right = .3cm of x2,draw] (x3) {2.7};
		\node [block, above = .3cm of x2] {\footnotesize Initial Conditions};
		\draw[-] (x1)--(x2);
		\draw[-] (x2)--(x3);
		
		\end{tikzpicture}
		\qquad	\begin{tikzpicture}[>=latex']
		\tikzset{block/.style= { rectangle, align=left,minimum width=.2cm,minimum height=.1cm},
			rblock/.style={draw, shape=rectangle,rounded corners=1.5em,align=center,minimum width=.2cm,minimum height=.1cm},
			input/.style={ % requires library shapes.geometric
				draw,
				trapezium,
				trapezium left angle=60,
				trapezium right angle=120,
				minimum width=2cm,
				align=center,
				minimum height=1cm
			},
		}
		\node [block,draw]  (x1) {3.2};
		\node [block, right = .3cm of x1,draw] (x2) {8.4};
		\node [block, right = .3cm of x2,draw] (x3) {3.2};
		\node [block, above = .3cm of x2] {\footnotesize After 2 Months};
		\draw[-] (x1)--(x2);
		\draw[-] (x2)--(x3);
		\end{tikzpicture}
		\qquad
		\begin{tikzpicture}[>=latex']
		\tikzset{block/.style= { rectangle, align=left,minimum width=.2cm,minimum height=.1cm},
			rblock/.style={draw, shape=rectangle,rounded corners=1.5em,align=center,minimum width=.2cm,minimum height=.1cm},
			input/.style={ % requires library shapes.geometric
				draw,
				trapezium,
				trapezium left angle=60,
				trapezium right angle=120,
				minimum width=2cm,
				align=center,
				minimum height=1cm
			},
		}
		\node [block,draw]  (x1) {4.3};
		\node [block, right = .3cm of x1,draw] (x2) {6.3};
		\node [block, right = .3cm of x2,draw] (x3) {4.3};
		\node [block, above = .3cm of x2] {\footnotesize After 7 Months};
		\draw[-] (x1)--(x2);
		\draw[-] (x2)--(x3);
		\end{tikzpicture}
	\end{center}
	
\end{frame}



\begin{frame}
	\frametitle{Implications of Estimates}
	\begin{itemize}
		\item SCSEM suggests both spatial and self-excitation present in model
		
		\begin{itemize}
			\item Mitigate local actors
			\item Identify and address root causes of spatial correlation
		\end{itemize}
		\item RDSEM suggests limited self-excitement, $\beta$ parameter suggests spread in space-time
		
		\begin{itemize}
			\item Isolate population
			\item Potentially due to coalition actions
		\end{itemize}
		\pause
		\item Choice of process model changes how terrorism is addressed
	\end{itemize}
\end{frame}




\section{Assessment and Selection}


\begin{frame}
	\frametitle{Model Assessment and Selection}
	\begin{itemize}
		\item Model selection via DIC
		\item Spiegelhalter et al (2002) provide estimate of effective number of parameters
		\begin{equation}
		pd(\boldsymbol{\theta}) \approx n-\text{tr}\left(\boldsymbol{Q(\theta)}\boldsymbol{Q^*(\theta)}^{-1}\right)
		\end{equation}
		\item Average $pd(\boldsymbol{\theta})$ over $\pi(\boldsymbol{\theta}|\boldsymbol{Y})$
		\item Deviance of mean is found by taking the expectation of the latent state after fixing $\boldsymbol{\theta}$ at its posterior mode where $\boldsymbol{X}$ is assumed to follow Gaussian distribution
		\begin{equation}
		-2\sum_{s_i,t} \log \pi\left(Y(s_i,t)|E[X(s_i,t)|\boldsymbol{\theta^*}],\boldsymbol{\theta^*}\right)
		\end{equation}
		
	\end{itemize}
\end{frame}
\begin{frame}
	\frametitle{Model Assessment and Selection}
	\begin{itemize}
		\item Model assessment using posterior predictive P values (Gelman et al, 1996)
		\item Assesses plausibility of model by highlighting discrepancies between model and data
		\item Pick ancillary statistic, $T(.)$ and calculate $T(\boldsymbol{Y})$
		\begin{itemize}
			\item for  $m=1...M$, draw a value of $\boldsymbol{\theta_m}$ according to $\pi(\boldsymbol{\theta}|\boldsymbol{Y})$
			\item Simulate $Y^*(\boldsymbol{s_i,t})_m$ of the same dimension as $\boldsymbol{Y}$ and compute $T(\boldsymbol{Y}^*_m)$
			\item Compute $\frac{1}{M}\sum_{m=1}^M I\left[T(\boldsymbol{Y}^*_m) > T(\boldsymbol{Y}) \right]$
		\end{itemize}
		
	\end{itemize}
\end{frame}

\begin{frame}
	\frametitle{Results and Conclusions}
	\begin{itemize}
		\item{DIC}
		\begin{itemize}
			\item SCSEM No Spatial - 10833
			\item SCSEM No Self Excitation - 10865
			\item SCSEM - 10302
			\item RDSEM - 8967
		\end{itemize}
		\item {Posterior Predictive Checks}
		\begin{itemize}
			\item Number of Zeroes
			\begin{itemize}
				\item SCSEM - 0
				\item RDSEM - .60
			\end{itemize}
			\item Maximum count
			\begin{itemize}
				\item SCSEM - .97
				\item RDSEM - .98
			\end{itemize}
		\end{itemize}
		\item All have trouble replicating maximum count - too high (sometimes significantly so)
		\item Some Evidence RDSEM fits data better - would suggest isolation or analyze US actions
	\end{itemize}
	
\end{frame}

\section{Dissertation Proposal/Way Ahead}


\begin{frame}
	\frametitle{Dissertation Proposal}
	\begin{itemize}
		\item Research Chapter 2 - Incorporating global events (Local Excitation due to common exogeneous events)
		\begin{itemize}
			\item Abu Ghraib, Golden Mosque bombing
		\end{itemize}
		\begin{align*}
		& Y(\boldsymbol{s_i},t)|\mu(\boldsymbol{s_i},t) \sim \text{Pois }(\mu(\boldsymbol{s_i},t)) \\
		&\small \mu(\boldsymbol{s_i},t) = \exp(X(\boldsymbol{s_i},t)) + \eta Y(\boldsymbol{s_i},t-1) + \frac{\kappa^{\alpha}}{\Gamma(\alpha)}(t-t_0)^{\alpha-1}\exp(-\kappa(t-t_0))\nonumber \\
		& t_0 = \text{ Time Global Shock Occurs}
		\end{align*}
		\pause
		\item Research Chapter 3 - Multiple actor Mutually-Exciting Spatio-Temporal Models
		\begin{align*}
		& Y(\boldsymbol{s_i},t)|\mu_1(\boldsymbol{s_i},t) \sim \text{Pois }(\mu_1(\boldsymbol{s_i},t)) \\
		& \mu_1(\boldsymbol{s_i},t) = \exp(X_1(\boldsymbol{s_i},t)) + \eta_1 Y(\boldsymbol{s_i},t-1) + \gamma_1 U(\boldsymbol{s_i},t-1) \\
		& U(\boldsymbol{s_i},t)|\mu_2(\boldsymbol{s_i},t) \sim \text{Pois }(\mu_2(\boldsymbol{s_i},t)) \\
		& \mu_2(\boldsymbol{s_i},t) = \exp(X_2(\boldsymbol{s_i},t)) + \eta_2 Y(\boldsymbol{s_i},t-1) + \gamma_2 U(\boldsymbol{s_i},t-1)\nonumber \\
		\end{align*}
		
	\end{itemize}
\end{frame}

\begin{frame}
	\frametitle{Proposed Timeline}
	\begin{itemize}
		\item Graduate - Spring 2018
		\item Defend Dissertation - April 2018
		\item Finish Chapter 3 - February 2018
		\item Present Chapter 1 at JSM - July 2017
		\item Finish Chapter 2 / Start Chapter 3 - July 2017
		\item Finish Chapter 1 / Submit to manuscript to Annals of Applied Statistics - March 2017
	\end{itemize}
	Thanks for the help and support!
\end{frame}




