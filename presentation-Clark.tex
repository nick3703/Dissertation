%%%%%%%%%%%%%%%%%%%%%%%%%%%%%%%%%%%%%%%%%
% Beamer Presentation
% LaTeX Template
% Version 1.0 (10/11/12)
%
% This template has been downloaded from:
% http://www.LaTeXTemplates.com
%
% License:
% CC BY-NC-SA 3.0 (http://creativecommons.org/licenses/by-nc-sa/3.0/)
%
%%%%%%%%%%%%%%%%%%%%%%%%%%%%%%%%%%%%%%%%%

%----------------------------------------------------------------------------------------
%	PACKAGES AND THEMES
%----------------------------------------------------------------------------------------

\documentclass{beamer}
\usepackage{etex}
\reserveinserts{28}
\mode<presentation> {

% The Beamer class comes with a number of default slide themes
% which change the colors and layouts of slides. Below this is a list
% of all the themes, uncomment each in turn to see what they look like.

%\usetheme{default}
%\usetheme{AnnArbor}
%\usetheme{Antibes}
%\usetheme{Bergen}
%\usetheme{Berkeley}
%\usetheme{Berlin}
%\usetheme{Boadilla}
%\usetheme{CambridgeUS}
%\usetheme{Copenhagen}
%\usetheme{Darmstadt}
%\usetheme{Dresden}
%\usetheme{Frankfurt}
%\usetheme{Goettingen}
%\usetheme{Hannover}
%\usetheme{Ilmenau}
%\usetheme{JuanLesPins}
%\usetheme{Luebeck}
\usetheme{Madrid}
%\usetheme{Malmoe}
%\usetheme{Marburg}
%\usetheme{Montpellier}
%\usetheme{PaloAlto}
%\usetheme{Pittsburgh}
%\usetheme{Rochester}
%\usetheme{Singapore}
%\usetheme{Szeged}
%\usetheme{Warsaw}

% As well as themes, the Beamer class has a number of color themes
% for any slide theme. Uncomment each of these in turn to see how it
% changes the colors of your current slide theme.

%\usecolortheme{albatross}
%\usecolortheme{beaver}
%\usecolortheme{beetle}
%\usecolortheme{crane}
%\usecolortheme{dolphin}
%\usecolortheme{dove}
%\usecolortheme{fly}
%\usecolortheme{lily}
%\usecolortheme{orchid}
%\usecolortheme{rose}
%\usecolortheme{seagull}
%\usecolortheme{seahorse}
%\usecolortheme{whale}
%\usecolortheme{wolverine}

%\setbeamertemplate{footline} % To remove the footer line in all slides uncomment this line
%\setbeamertemplate{footline}[page number] % To replace the footer line in all slides with a simple slide count uncomment this line

%\setbeamertemplate{navigation symbols}{} % To remove the navigation symbols from the bottom of all slides uncomment this line
}




\usepackage{graphicx} % Allows including images
\usepackage{booktabs} % Allows the use of \toprule, \midrule and \bottomrule in tables
\usepackage{amsmath}
\usepackage{verbatim}
\usepackage{tikz}
\usepackage{amsmath,amsfonts} % Math packages
\usepackage{listings}
\usepackage{appendixnumberbeamer}
\usepackage{xcolor}
\usepackage{amsthm}
\usepackage{algorithm,algorithmic}
\makeatletter
\newenvironment<>{proofs}[1][\proofname]{%
	\par
	\def\insertproofname{#1\@addpunct{.}}%
	\usebeamertemplate{proof begin}#2}
{\usebeamertemplate{proof end}}
\makeatother
\setbeamercolor{section in toc}{fg=black}
\setbeamercolor{subsection in toc}{fg=black}


\newcommand{\highlight}[1]{%
	\colorbox{orange!50}{$\displaystyle#1$}}


\newcommand{\highlighter}[1]{%
	\colorbox{blue!30}{$\displaystyle#1$}}

\usetikzlibrary{arrows,chains,matrix,positioning,scopes,fit}
%
\makeatletter
\tikzset{join/.code=\tikzset{after node path={%
			\ifx\tikzchainprevious\pgfutil@empty\else(\tikzchainprevious)%
			edge[every join]#1(\tikzchaincurrent)\fi}}}
\makeatother
%
\tikzset{>=stealth',every on chain/.append style={join},
	every join/.style={->}}
\tikzstyle{labeled}=[execute at begin node=$\scriptstyle,
execute at end node=$]

\usetikzlibrary{arrows,positioning,shapes.geometric}

%

\usepackage[latin1]{inputenc}
\usepackage{times}
\usepackage{tikz}

\usepackage{empheq}
\usepackage[most]{tcolorbox}

\newtcbox{\othermathbox}[1][]{nobeforeafter, math upper, tcbox raise base, 
	enhanced, sharp corners, colback=black!10, colframe=red!30!black, 
	drop fuzzy shadow, left=1em, top=1em, right=1em, bottom=1em}

\setbeamertemplate{caption}{\raggedright\insertcaption\par}


%----------------------------------------------------------------------------------------
%	TITLE PAGE
%----------------------------------------------------------------------------------------

\title[Self-Exciting Spatio-Temporal Models]{Self-Exciting Spatio-Temporal Models for Count Data} % The short title appears at the bottom of every slide, the full title is only on the title page

\author{Nicholas Clark} % Your name
\institute[ISU] % Your institution as it will appear on the bottom of every slide, may be shorthand to save space
{
Iowa State University \\ % Your institution for the title page
\medskip
\textit{nclark1@iastate.edu} % Your email address
}
\date{\today} % Date, can be changed to a custom date

\begin{document}

\begin{frame}
\titlepage % Print the title page as the first slide
\end{frame}

%\begin{frame}
%\frametitle{Overview} % Table of contents slide, comment this block out to remove it
%\tableofcontents % Throughout your presentation, if you choose to use \section{} and \subsection{} commands, these will automatically be printed on this slide as an overview of your presentation
%\end{frame}

%----------------------------------------------------------------------------------------
%	PRESENTATION SLIDES
%----------------------------------------------------------------------------------------



\begin{frame}
	\frametitle{Motivation: The Evolution of Violence in Space and Time}
\textit{At present, the most under-researched area of spatial criminology is that of spatio-temporal crime patterns...  the temporal component of the underlying crime distributions has languished as a largely ignored area of study} - \textbf{Crime mapping: Spatial and Temporal Challenges}, Ratcliffe (2010)

\end{frame}

\begin{frame}
	\frametitle{The Spread of Violence in Iraq 2004}
	
	\begin{figure}[ht]
		\centering
		\includegraphics[width=8cm]{Iraq2}
		\label{fig:SCSEMExample}
	\end{figure}

\end{frame}		

\begin{frame}
	\frametitle{Burglaries South Side of Chicago}
	\begin{figure}[ht]
		\centering
		\includegraphics[width=8cm]{ChiBurg}
		\label{fig:SCSEMExample}
	\end{figure}
\end{frame}		

\begin{frame}
	\frametitle{Goals}
	\begin{itemize}
		
		\item General statistical model for diffusion of violence in space-time
		
			\begin{itemize}
		\item Accurately reflects beliefs on how violence/crime evolves
		\newline
		\item Extends traditional statistical models for count data
		\newline
		\item Stationary with extremely flexible second order properties
		\newline
		\item Inference via traditional MCMC techniques 
		\end{itemize}
		\end{itemize}
\end{frame}	

\begin{frame}
\frametitle{Overview} % Table of contents slide, comment this block out to remove it
\tableofcontents % Throughout your presentation, if you choose to use \section{} and \subsection{} commands, these will automatically be printed on this slide as an overview of your presentation
\end{frame}

\section{Mathematical Model for Diffusion of Crime and Related Statistical Models}

\begin{frame}
	\frametitle{A Model of Criminal Behavior (Short et al. 2008)}
		\begin{itemize}
			\item $Z(s_i,t)$ - number of observed burglaries from $(t-\Delta t,t)$
			\item $s_i \in \{s_1,\cdots,s_{n_d}\}$ - fixed regions in $\mathbb{R}^2$
			\item $t \in \{1,\cdots,T\}$ - discrete time
			\item  Define $A(s_i,t)\equiv A(s_i,0)+B(s_i,t)$ as attractiveness
			\begin{equation}
			B(s_i,t)=\left(1-\chi \Delta t\right) B(s_i,t- \Delta t)+\eta Z(s_i,t- \Delta t)
			\end{equation}
			\item $Z(s_i,t) \sim \mbox{Pois }(A(s_i,t))$
			\item Three factors impact change in crime rate, base attractiveness , decay  $\chi$, and repeat victimization, $\eta$
		\end{itemize}
\end{frame}

 \begin{frame}
 	\frametitle{Relationship to INGARCH Model}
 	Integer Auto-Regressive Conditionally Heteroskedastic, INGARCH (1,1), or Poisson Auto-Regression Model
 	\begin{align}
 	Z(s_i,t) & \sim \mbox{Pois }(\lambda(s_i,t))\\
 	\lambda(s_i,t)&= d+ a\lambda(s_i,t-1)+b Z(s_i,t-1)
 	\end{align}
 	\begin{itemize}
 	\item Unlike GARCH, not solely a variance property
	 \item Short model is similar to INGARCH(1,1) with $A(s_i,0)=\sum_{k=0}^t a^k d$, $a=(1-\chi \Delta t)$, and $b=\eta$
	 \end{itemize}
 \end{frame}

\begin{frame}
	\frametitle{Relationship to Self-Exciting Models}
	\begin{itemize}
		\item Point process introduced by Alan Hawkes with intensity
		\begin{align}
		& \lambda(t)= \nu(t) + \int_{0}^{t} g (t-u) N(ds)
		\end{align}
		\item Commonly discretized as
			\begin{align}
			& Z(s_i,t)\sim \mbox{Pois }(\lambda(s_i,t))\\
			& \lambda(s_i,t) = \nu + \sum_{j < t} \eta^{t-j}Z(s_i,t-j)
			\end{align}
		\item Equivalent to stationary INGARCH(1,1)
	\end{itemize}
\end{frame}


\begin{frame}
		\frametitle{Structural Diagram - INGARCH(1,1)}
	\begin{figure}
		\centering
		
		\begin{tikzpicture}[>=latex']
		\tikzset{block/.style= { rectangle, align=left,minimum width=.2cm,minimum height=.1cm},
			rblock/.style={draw, shape=rectangle,rounded corners=1.5em,align=center,minimum width=.2cm,minimum height=.1cm},
			input/.style={ % requires library shapes.geometric
				draw,
				trapezium,
				trapezium left angle=60,
				trapezium right angle=120,
				minimum width=2cm,
				align=center,
				minimum height=1cm
			},
		}
		\node [block]  (Y1) {\footnotesize $d$};
		\node [block, right = 2.5cm of Y1] (Y2) {\footnotesize $d$};
		\node [block, right = 2.5cm of Y2] (Y3) {\footnotesize $d$};
		\node [block, below = .5cm of Y1] (lam1) {\footnotesize $\lambda(s_1,1)$};
		\node [block, below = .5cm of Y2] (lam2) {\footnotesize $\lambda(s_2,1)$};
		\node [block, below = .5cm of Y3] (lam3) {\footnotesize $\lambda(s_3,1)$};
		\node [block, below = .5cm of lam1] (Z1) {\footnotesize $Z(s_1,1)$};
		\node [block, below = .5cm of lam2] (Z2) {\footnotesize $Z(s_2,1)$};
		\node [block, below = .5cm of lam3] (Z3) {\footnotesize $Z(s_3,1)$};
		\node [block, below = .5cm of Z1] (Z21) {\footnotesize $Z(s_1,2)$};
		\node [block, below = .5cm of Z2] (Z22) {\footnotesize $Z(s_2,2)$};
		\node [block, below = .5cm of Z3] (Z23) {\footnotesize $Z(s_3,2)$};
		\node [block, below = .5cm of Z21] (lam21) {\footnotesize $\lambda(s_1,2)$};
		\node [block, below = .5cm of Z22] (lam22) {\footnotesize $\lambda(s_2,2)$};
		\node [block, below = .5cm of Z23] (lam23) {\footnotesize $\lambda(s_3,2)$};
		\node [block, below = .5cm of lam21] (Y21) {\footnotesize $d$};
		\node [block, below = .5cm of lam22] (Y22) {\footnotesize $d$};
		\node [block, below = .5cm of lam23] (Y23) {\footnotesize $d$};
		\draw [->] (Y1) edge[line width=.4mm] (lam1);
		\draw [->] (Y2) edge[line width=.4mm] (lam2);
		\draw [->] (Y3) edge[line width=.4mm] (lam3);
		\draw [->] (lam1) edge[line width=.4mm] (Z1);
		\draw [->] (lam2) edge[line width=.4mm] (Z2);
		\draw [->] (lam3) edge[line width=.4mm] (Z3);
		\draw [->] (Y21) edge[line width=.4mm] (lam21);
		\draw [->] (Y22) edge[line width=.4mm] (lam22);
		\draw [->] (Y23) edge[line width=.4mm] (lam23);
		\draw [->] (lam21) edge[line width=.4mm] (Z21);
		\draw [->] (lam22) edge[line width=.4mm] (Z22);
		\draw [->] (lam23) edge[line width=.4mm] (Z23);
			\node[label={[red]above:{Time Period 1}},draw=red, fit=(Y1) (Y2) (Y3) (lam1) (lam2) (lam3) (Z1) (Z2) (Z3)](Fit1) {};
			\node[label={[blue]below:{Time Period 2}},draw=blue, fit=(Y21) (Y22) (Y23) (lam21) (lam22) (lam23) (Z21) (Z22) (Z23)](Fit1) {};
		\draw[->] (lam1) edge[bend right=50,line width=.5mm] (lam21);
		\draw[->] (lam2) edge[bend right=50,line width=.5mm] (lam22);
		\draw[->] (lam3) edge[bend right=50,line width=.5mm] (lam23);
		\draw[->] (Z1) edge[bend left=50,line width=.5mm] (lam21);
		\draw[->] (Z2) edge[bend left=50,line width=.5mm] (lam22);
		\draw[->] (Z3) edge[bend left=50,line width=.5mm] (lam23);
	
		\end{tikzpicture}
	\end{figure}
\end{frame}

\begin{frame}
	\frametitle{Short (2008) Extension - Spatial Spread}
	
		\begin{itemize}
			\item Motivated by Reaction-Diffusion PDE
			\item  Recall $A(s_i,t)\equiv A(s_i,0)+B(s_i,t)$ as attractiveness
			\item Define: $N_i=\{s_j :s_j\text{ is a spatial neighbor of } s_i\}$, $|N_{s_i}|$ is number of neighbors
			\begin{align}
			B(s_i,t)& =\kappa B(s_i,t- \Delta t) + \frac{\psi}{|N_{s_i}|}\sum_{s_j \in N_{i}} \left[B(s_j,t- \Delta t)-B(s_i,t- \Delta t)\right]\nonumber\\
			& +\eta Z(s_i,t- \Delta t)
			\end{align}
			\item Four factors impact change in crime rate, base attractiveness, decay  $\chi$, and repeat victimization, $\eta$, and spatial spread $\psi$
			\item Resulting model is MINGARCH (1,1)
		\end{itemize}
	
\end{frame}


\begin{frame}
	\frametitle{Short (2008) Extension - Spatial Spread}
	\begin{figure}
		\centering
		
		\begin{tikzpicture}[>=latex']
		\tikzset{block/.style= { rectangle, align=left,minimum width=.2cm,minimum height=.1cm},
			rblock/.style={draw, shape=rectangle,rounded corners=1.5em,align=center,minimum width=.2cm,minimum height=.1cm},
			input/.style={ % requires library shapes.geometric
				draw,
				trapezium,
				trapezium left angle=60,
				trapezium right angle=120,
				minimum width=2cm,
				align=center,
				minimum height=1cm
			},
		}
		\node [block]  (Y1) {\footnotesize $d$};
		\node [block, right = 2.5cm of Y1] (Y2) {\footnotesize $d$};
		\node [block, right = 2.5cm of Y2] (Y3) {\footnotesize $d$};
		\node [block, below = .5cm of Y1] (lam1) {\footnotesize $\lambda(s_1,1)$};
		\node [block, below = .5cm of Y2] (lam2) {\footnotesize $\lambda(s_2,1)$};
		\node [block, below = .5cm of Y3] (lam3) {\footnotesize $\lambda(s_3,1)$};
		\node [block, below = .5cm of lam1] (Z1) {\footnotesize $Z(s_1,1)$};
		\node [block, below = .5cm of lam2] (Z2) {\footnotesize $Z(s_2,1)$};
		\node [block, below = .5cm of lam3] (Z3) {\footnotesize $Z(s_3,1)$};
		\node [block, below = .5cm of Z1] (Z21) {\footnotesize $Z(s_1,2)$};
		\node [block, below = .5cm of Z2] (Z22) {\footnotesize $Z(s_2,2)$};
		\node [block, below = .5cm of Z3] (Z23) {\footnotesize $Z(s_3,2)$};
		\node [block, below = .5cm of Z21] (lam21) {\footnotesize $\lambda(s_1,2)$};
		\node [block, below = .5cm of Z22] (lam22) {\footnotesize $\lambda(s_2,2)$};
		\node [block, below = .5cm of Z23] (lam23) {\footnotesize $\lambda(s_3,2)$};
		\node [block, below = .5cm of lam21] (Y21) {\footnotesize $d$};
		\node [block, below = .5cm of lam22] (Y22) {\footnotesize $d$};
		\node [block, below = .5cm of lam23] (Y23) {\footnotesize $d$};
		\draw [->] (Y1) edge[line width=.4mm] (lam1);
		\draw [->] (Y2) edge[line width=.4mm] (lam2);
		\draw [->] (Y3) edge[line width=.4mm] (lam3);
		\draw [->] (lam1) edge[line width=.4mm] (Z1);
		\draw [->] (lam2) edge[line width=.4mm] (Z2);
		\draw [->] (lam3) edge[line width=.4mm] (Z3);
		\draw [->] (Y21) edge[line width=.4mm] (lam21);
		\draw [->] (Y22) edge[line width=.4mm] (lam22);
		\draw [->] (Y23) edge[line width=.4mm] (lam23);
		\draw [->] (lam21) edge[line width=.4mm] (Z21);
		\draw [->] (lam22) edge[line width=.4mm] (Z22);
		\draw [->] (lam23) edge[line width=.4mm] (Z23);
		\node[label={[red]above:{Time Period 1}},draw=red, fit=(Y1) (Y2) (Y3) (lam1) (lam2) (lam3) (Z1) (Z2) (Z3)](Fit1) {};
		\node[label={[blue]below:{Time Period 2}},draw=blue, fit=(Y21) (Y22) (Y23) (lam21) (lam22) (lam23) (Z21) (Z22) (Z23)](Fit1) {};
		\draw[->] (lam1) edge[bend right=50,line width=.5mm] (lam21);
		\draw[->] (lam1) edge[bend left=-20,line width=.5mm,color=red] (lam22);
		\draw[->] (lam2) edge[bend right=50,line width=.5mm] (lam22);
		\draw[->] (lam3) edge[bend right=50,line width=.5mm] (lam23);
		\draw[->] (lam3) edge[bend right=-10,line width=.5mm,color=red] (lam22);
		\draw[->] (Z1) edge[bend left=50,line width=.5mm] (lam21);
		\draw[->] (Z2) edge[bend left=50,line width=.5mm] (lam22);
		\draw[->] (Z3) edge[bend left=50,line width=.5mm] (lam23);
		
		\end{tikzpicture}
	\end{figure}
\end{frame}

\begin{frame}
	\frametitle{Applied to Residential Burglaries in Chicago}
	\begin{figure}[ht]
		\centering
		\includegraphics[width=8cm]{ChiBurg}
	\end{figure}
	552 Spatial Locations, 72 Months, residential burglaries
\end{frame}	

 \begin{frame}
 	\frametitle{Applied to Residential Burglaries in Chicago}
 	\begin{align}
 	& B(s_i,t) =\kappa B(s_i,t- \Delta t) \nonumber\\&+ \frac{\psi}{|N_{s_i}|}\sum_{s_j \in N_{s_i}} \left[B(s_j,t- \Delta t)-B(s_i,t- \Delta t)\right]
 	+\eta Z(s_i,t- \Delta t)
 	\end{align}
 	\begin{itemize}
 	\item Further structure $A(s_i,0)$ to account for socio-economic factors
  \item MLEs are $\hat{\psi}=.02$, $\hat{\eta}=.23$ $\hat{\kappa}=.461$
 	
 	\item Simulate data from asymptotic distribution - unable to replicate lag-one autocorrelation, spatial correlation, or variance to mean ratio of original data
 	\end{itemize}
\end{frame}

    \subsection{Issues with Model}
    \begin{frame}
    	\frametitle{Properties of INGARCH Model}
    	\begin{empheq}[box=\othermathbox]{align}
    		Z(s_i,t) & \sim \mbox{Pois }(\lambda(s_i,t))\nonumber\\
    		\lambda(s_i,t)&= d+ a\lambda(s_i,t-1)+b Z(s_i,t-1)\nonumber
    	\end{empheq}
    	
    	Stationarity yields:
    	\begin{align}
    	& E[Z(s_i,t)] = \frac{d}{1-(a+b)}\\
    	& \mbox{Var}[Z(s_i,t)] = \frac{1-(a+b)^2+b^2}{1-(a+b)^2} E[Z(s_i,t)] \\
    	& \mbox{Cov} [Z(s_i,t),Z(s_i,t-h)] = \frac{b(1-a(a+b))(a+b)^h}{1-(a+b)^2} E[Z(s_i,t)]\\
    	& \mbox{Var-Mean Ratio} [Z(s_i,t)] = 1+\frac{b^2}{1-(a+b)^2}
    	\end{align}
    	
    \end{frame}
    
    \begin{frame}
    	\frametitle{Issues}
    	Allows for Overdispersion...  But at a cost!
    	\begin{align}
    	& \mbox{Cor} [Z(s_i,t),Z(s_i,t-1)] = \frac{b(a+b)(a^2+ab-1)}{a^2+2ab-1}\\
    	& \mbox{Var-Mean Ratio} [Z(s_i,t)] = 1+\frac{b^2}{1-(a+b)^2}\\
    	\end{align}
    	For fixed Var-Mean Ratio at 2 $\implies b=1/2 (-a +\sqrt{2-a^2})$. 
    	\begin{figure}[t!]
    		\centering
    		\includegraphics[width=0.4\textwidth]{INGARCHIssue}
    	\end{figure}
    \end{frame}



 \begin{frame}
 	\frametitle{\small Spatially Correlated Self-Exciting Model (Clark \& Dixon, 2018)}
 	$Y(s_i,t)$ - Spatially Correlated Latent Gaussian
 	
 	\begin{figure}
 		\centering
 		
 		\begin{tikzpicture}[>=latex']
 		\tikzset{block/.style= { rectangle, align=left,minimum width=.2cm,minimum height=.1cm},
 			rblock/.style={draw, shape=rectangle,rounded corners=1.5em,align=center,minimum width=.2cm,minimum height=.1cm},
 			input/.style={ % requires library shapes.geometric
 				draw,
 				trapezium,
 				trapezium left angle=60,
 				trapezium right angle=120,
 				minimum width=2cm,
 				align=center,
 				minimum height=1cm
 			},
 		}
 		\node [block]  (Y1) {\footnotesize $\exp(Y(s_1,1))$};
 		\node [block, right = .5cm of Y1] (Y2) {\footnotesize $\exp(Y(s_2,1))$};
 		\node [block, right = .5cm of Y2] (Y3) {\footnotesize $\exp(Y(s_3,1))$};
 		\node [block, below = .5cm of Y1] (lam1) {\footnotesize $\lambda(s_1,1)$};
 		\node [block, below = .5cm of Y2] (lam2) {\footnotesize $\lambda(s_2,1)$};
 		\node [block, below = .5cm of Y3] (lam3) {\footnotesize $\lambda(s_3,1)$};
 		\node [block, below = .5cm of lam1] (Z1) {\footnotesize $Z(s_1,1)$};
 		\node [block, below = .5cm of lam2] (Z2) {\footnotesize $Z(s_2,1)$};
 		\node [block, below = .5cm of lam3] (Z3) {\footnotesize $Z(s_3,1)$};
 		\node [block, below = .5cm of Z1] (Z21) {\footnotesize $Z(s_1,2)$};
 		\node [block, below = .5cm of Z2] (Z22) {\footnotesize $Z(s_2,2)$};
 		\node [block, below = .5cm of Z3] (Z23) {\footnotesize $Z(s_3,2)$};
 		\node [block, below = .5cm of Z21] (lam21) {\footnotesize $\lambda(s_1,2)$};
 		\node [block, below = .5cm of Z22] (lam22) {\footnotesize $\lambda(s_2,2)$};
 		\node [block, below = .5cm of Z23] (lam23) {\footnotesize $\lambda(s_3,2)$};
 		\node [block, below = .5cm of lam21] (Y21) {\footnotesize $\exp(Y(s_1,2))$};
 		\node [block, below = .5cm of lam22] (Y22) {\footnotesize $\exp(Y(s_2,2))$};
 		\node [block, below = .5cm of lam23] (Y23) {\footnotesize $\exp(Y(s_3,2))$};
 		\draw [<->] (Y1) edge[line width=.4mm] (Y2);
 		\draw [<->] (Y2) edge[line width=.4mm] (Y3);
 		\draw [->] (Y1) edge[line width=.4mm] (lam1);
 		\draw [->] (Y2) edge[line width=.4mm] (lam2);
 		\draw [->] (Y3) edge[line width=.4mm] (lam3);
 		\draw [->] (lam1) edge[line width=.4mm] (Z1);
 		\draw [->] (lam2) edge[line width=.4mm] (Z2);
 		\draw [->] (lam3) edge[line width=.4mm] (Z3);
 		\draw [<->] (Y21) edge[line width=.4mm] (Y22);
 		\draw [<->] (Y22) edge[line width=.4mm] (Y23);
 		\draw [->] (Y21) edge[line width=.4mm] (lam21);
 		\draw [->] (Y22) edge[line width=.4mm] (lam22);
 		\draw [->] (Y23) edge[line width=.4mm] (lam23);
 		\draw [->] (lam21) edge[line width=.4mm] (Z21);
 		\draw [->] (lam22) edge[line width=.4mm] (Z22);
 		\draw [->] (lam23) edge[line width=.4mm] (Z23);
 		
 		\node[label={[red]above:{Time Period 1}},draw=red, fit=(Y1) (Y2) (Y3) (lam1) (lam2) (lam3) (Z1) (Z2) (Z3)](Fit1) {};
 		\node[label={[blue]below:{Time Period 2}},draw=blue, fit=(Y21) (Y22) (Y23) (lam21) (lam22) (lam23) (Z21) (Z22) (Z23)](Fit1) {};
 		\draw[->] (Z1) edge[bend right=50,line width=.5mm] (lam21);
 		\draw[->] (Z2) edge[bend right=50,line width=.5mm] (lam22);
 		\draw[->] (Z3) edge[bend right=50,line width=.5mm] (lam23);
 		\end{tikzpicture}
 		\caption{Structure Diagram for statistical model given in \eqref{eq:example}} \label{fig:structure1}
 	\end{figure}
 \end{frame}
\begin{frame}
	\frametitle{\small Spatially Correlated Self-Exciting Model (Clark \& Dixon, 2018)}
	\begin{itemize}
		\item \textbf{Theory:} Exist common cause between geographically similar locations, regions that experience uptick in violence likely to have short term self-excitation
		\pause
		\item Mixture of two processes that influence expectation : LGCP and Hawkes process
		\item Hawkes process letting $g(t-j)=\eta$ if $(t-j)=1$, $0$ otherwise
		\begin{align}
		& Z(\boldsymbol{s_i},t)|\lambda(\boldsymbol{s_i},t) \sim \text{Pois }(\lambda(\boldsymbol{s_i},t)) \label{eq:Full Model2}\\
		& \lambda(\boldsymbol{s_i},t) = \exp(Y(\boldsymbol{s_i},t)) + \eta Z(\boldsymbol{s_i},t-1) \nonumber \\
		& Y(\boldsymbol{s_i},t) = \theta_1 \sum_{\boldsymbol{s_j}\in N(\boldsymbol{s_i})}Y(\boldsymbol{s_j},t) + \epsilon(\boldsymbol{s_i},t) \nonumber\\
		&\epsilon(\boldsymbol{s_i},t) \sim Gau(0,\sigma^2) \nonumber
		\end{align}
		
		
	\end{itemize}
\end{frame}

%------------------------------------------------
% Sections can be created in order to organize your presentation into discrete blocks, all sections and subsections are automatically printed in the table of contents as an overview of the talk
%------------------------------------------------

 % A subsection can be created just before a set of slides with a common theme to further break down your presentation into chunks
 \begin{frame}
 	\frametitle{\small Spatially Correlated Self-Exciting Model (Clark \& Dixon, 2018)}
 	$Y(s_i,t)$ - Spatially Correlated Latent Gaussian
 	
 	\begin{figure}
 		\centering
 		
 		\begin{tikzpicture}[>=latex']
 		\tikzset{block/.style= { rectangle, align=left,minimum width=.2cm,minimum height=.1cm},
 			rblock/.style={draw, shape=rectangle,rounded corners=1.5em,align=center,minimum width=.2cm,minimum height=.1cm},
 			input/.style={ % requires library shapes.geometric
 				draw,
 				trapezium,
 				trapezium left angle=60,
 				trapezium right angle=120,
 				minimum width=2cm,
 				align=center,
 				minimum height=1cm
 			},
 		}
 		\node [block]  (Y1) {\footnotesize $\exp(Y(s_1,1))$};
 		\node [block, right = .5cm of Y1] (Y2) {\footnotesize $\exp(Y(s_2,1))$};
 		\node [block, right = .5cm of Y2] (Y3) {\footnotesize $\exp(Y(s_3,1))$};
 		\node [block, below = .5cm of Y1] (lam1) {\footnotesize $\lambda(s_1,1)$};
 		\node [block, below = .5cm of Y2] (lam2) {\footnotesize $\lambda(s_2,1)$};
 		\node [block, below = .5cm of Y3] (lam3) {\footnotesize $\lambda(s_3,1)$};
 		\node [block, below = .5cm of lam1] (Z1) {\footnotesize $Z(s_1,1)$};
 		\node [block, below = .5cm of lam2] (Z2) {\footnotesize $Z(s_2,1)$};
 		\node [block, below = .5cm of lam3] (Z3) {\footnotesize $Z(s_3,1)$};
 		\node [block, below = .5cm of Z1] (Z21) {\footnotesize $Z(s_1,2)$};
 		\node [block, below = .5cm of Z2] (Z22) {\footnotesize $Z(s_2,2)$};
 		\node [block, below = .5cm of Z3] (Z23) {\footnotesize $Z(s_3,2)$};
 		\node [block, below = .5cm of Z21] (lam21) {\footnotesize $\lambda(s_1,2)$};
 		\node [block, below = .5cm of Z22] (lam22) {\footnotesize $\lambda(s_2,2)$};
 		\node [block, below = .5cm of Z23] (lam23) {\footnotesize $\lambda(s_3,2)$};
 		\node [block, below = .5cm of lam21] (Y21) {\footnotesize $\exp(Y(s_1,2))$};
 		\node [block, below = .5cm of lam22] (Y22) {\footnotesize $\exp(Y(s_2,2))$};
 		\node [block, below = .5cm of lam23] (Y23) {\footnotesize $\exp(Y(s_3,2))$};
 		\draw [<->] (Y1) edge[line width=.4mm] (Y2);
 		\draw [<->] (Y2) edge[line width=.4mm] (Y3);
 		\draw [->] (Y1) edge[line width=.4mm] (lam1);
 		\draw [->] (Y2) edge[line width=.4mm] (lam2);
 		\draw [->] (Y3) edge[line width=.4mm] (lam3);
 		\draw [->] (lam1) edge[line width=.4mm] (Z1);
 		\draw [->] (lam2) edge[line width=.4mm] (Z2);
 		\draw [->] (lam3) edge[line width=.4mm] (Z3);
 		\draw [<->] (Y21) edge[line width=.4mm] (Y22);
 		\draw [<->] (Y22) edge[line width=.4mm] (Y23);
 		\draw [->] (Y21) edge[line width=.4mm] (lam21);
 		\draw [->] (Y22) edge[line width=.4mm] (lam22);
 		\draw [->] (Y23) edge[line width=.4mm] (lam23);
 		\draw [->] (lam21) edge[line width=.4mm] (Z21);
 		\draw [->] (lam22) edge[line width=.4mm] (Z22);
 		\draw [->] (lam23) edge[line width=.4mm] (Z23);
 	
 		\node[label={[red]above:{Time Period 1}},draw=red, fit=(Y1) (Y2) (Y3) (lam1) (lam2) (lam3) (Z1) (Z2) (Z3)](Fit1) {};
 		\node[label={[blue]below:{Time Period 2}},draw=blue, fit=(Y21) (Y22) (Y23) (lam21) (lam22) (lam23) (Z21) (Z22) (Z23)](Fit1) {};
 			\draw[->] (Z1) edge[bend right=50,line width=.5mm] (lam21);
 			\draw[->] (Z2) edge[bend right=50,line width=.5mm] (lam22);
 			\draw[->] (Z3) edge[bend right=50,line width=.5mm] (lam23);
 		\end{tikzpicture}
 		\caption{Structure Diagram for statistical model given in \eqref{eq:example}} \label{fig:structure1}
 	\end{figure}
 \end{frame}
 \section{SPINGARCH Model} 
 \begin{frame}
 	\frametitle{SPINGARCH(1,1) Model}
 	Spatially Correlated INGARCH(1,1) Model
 	
 	\begin{figure}
 		\centering
 		
 		\begin{tikzpicture}[>=latex']
 		\tikzset{block/.style= { rectangle, align=left,minimum width=.2cm,minimum height=.1cm},
 			rblock/.style={draw, shape=rectangle,rounded corners=1.5em,align=center,minimum width=.2cm,minimum height=.1cm},
 			input/.style={ % requires library shapes.geometric
 				draw,
 				trapezium,
 				trapezium left angle=60,
 				trapezium right angle=120,
 				minimum width=2cm,
 				align=center,
 				minimum height=1cm
 			},
 		}
 		\node [block]  (Y1) {\footnotesize $\exp(Y(s_1,1))$};
 		\node [block, right = .5cm of Y1] (Y2) {\footnotesize $\exp(Y(s_2,1))$};
 		\node [block, right = .5cm of Y2] (Y3) {\footnotesize $\exp(Y(s_3,1))$};
 		\node [block, below = .5cm of Y1] (lam1) {\footnotesize $\lambda(s_1,1)$};
 		\node [block, below = .5cm of Y2] (lam2) {\footnotesize $\lambda(s_2,1)$};
 		\node [block, below = .5cm of Y3] (lam3) {\footnotesize $\lambda(s_3,1)$};
 		\node [block, below = .5cm of lam1] (Z1) {\footnotesize $Z(s_1,1)$};
 		\node [block, below = .5cm of lam2] (Z2) {\footnotesize $Z(s_2,1)$};
 		\node [block, below = .5cm of lam3] (Z3) {\footnotesize $Z(s_3,1)$};
 		\node [block, below = .5cm of Z1] (Z21) {\footnotesize $Z(s_1,2)$};
 		\node [block, below = .5cm of Z2] (Z22) {\footnotesize $Z(s_2,2)$};
 		\node [block, below = .5cm of Z3] (Z23) {\footnotesize $Z(s_3,2)$};
 		\node [block, below = .5cm of Z21] (lam21) {\footnotesize $\lambda(s_1,2)$};
 		\node [block, below = .5cm of Z22] (lam22) {\footnotesize $\lambda(s_2,2)$};
 		\node [block, below = .5cm of Z23] (lam23) {\footnotesize $\lambda(s_3,2)$};
 		\node [block, below = .5cm of lam21] (Y21) {\footnotesize $\exp(Y(s_1,2))$};
 		\node [block, below = .5cm of lam22] (Y22) {\footnotesize $\exp(Y(s_2,2))$};
 		\node [block, below = .5cm of lam23] (Y23) {\footnotesize $\exp(Y(s_3,2))$};
 		\draw [<->] (Y1) edge[line width=.4mm] (Y2);
 		\draw [<->] (Y2) edge[line width=.4mm] (Y3);
 		\draw [->] (Y1) edge[line width=.4mm] (lam1);
 		\draw [->] (Y2) edge[line width=.4mm] (lam2);
 		\draw [->] (Y3) edge[line width=.4mm] (lam3);
 		\draw [->] (lam1) edge[line width=.4mm] (Z1);
 		\draw [->] (lam2) edge[line width=.4mm] (Z2);
 		\draw [->] (lam3) edge[line width=.4mm] (Z3);
 		\draw [<->] (Y21) edge[line width=.4mm] (Y22);
 		\draw [<->] (Y22) edge[line width=.4mm] (Y23);
 		\draw [->] (Y21) edge[line width=.4mm] (lam21);
 		\draw [->] (Y22) edge[line width=.4mm] (lam22);
 		\draw [->] (Y23) edge[line width=.4mm] (lam23);
 		\draw [->] (lam21) edge[line width=.4mm] (Z21);
 		\draw [->] (lam22) edge[line width=.4mm] (Z22);
 		\draw [->] (lam23) edge[line width=.4mm] (Z23);
 		\node[label={[red]above:{Time Period 1}},draw=red, fit=(Y1) (Y2) (Y3) (lam1) (lam2) (lam3) (Z1) (Z2) (Z3)](Fit1) {};
 		\node[label={[blue]below:{Time Period 2}},draw=blue, fit=(Y21) (Y22) (Y23) (lam21) (lam22) (lam23) (Z21) (Z22) (Z23)](Fit1) {};
 			\draw[->] (lam1) edge[bend right=50,line width=.5mm] (lam21);
 			\draw[->] (lam2) edge[bend right=50,line width=.5mm] (lam22);
 			\draw[->] (lam3) edge[bend right=50,line width=.5mm] (lam23);
 			\draw[->] (Z1) edge[bend left=50,line width=.5mm] (lam21);
 			\draw[->] (Z2) edge[bend left=50,line width=.5mm] (lam22);
 			\draw[->] (Z3) edge[bend left=50,line width=.5mm] (lam23);
 		\end{tikzpicture}
 		\end{figure}
 \end{frame}
 
  \begin{frame}
  	\frametitle{SPINGARCH Model}
  	\textbf{Theory:} Exist common cause between geographically similar locations, regions that experience uptick in violence likely to have short term self-excitation, \textbf{\textit{absence of violence or exogeneous effects reduces tension}}
  	\pause
  	\begin{itemize}
  	\item $Z(s_i,t)|Y(s_i,t),\mathcal{H}_{Z(s_i)}\sim\mbox{Pois }(\lambda(s_i,t))$ where $\mathcal{H}_{Z(s_i)}$ is history of process at location $s_i$
  	\begin{align}
	&\lambda(s_i,t)=\exp \left[ Y(s_i,t) \right] + \eta Z(s_i,t-1) + \kappa \lambda(s_i,t-1)\nonumber\\
  	\end{align}
\pause
  	\begin{align}
  	& Y(s_i,t)|\boldsymbol{Y}(N_i)\sim N(\mu(s_i,t),\sigma_{sp}^2) \label{eq:Latent Dependency}\\
  	& \mu(s_i,t) = \alpha(s_i)+ \zeta \sum_{s_j \in N_i} \{Y(s_j,t)-\alpha(s_j)\} \nonumber.
  	\end{align}
  	\pause
  	\item $\eta=0,\kappa=0$ Poisson - CAR, $\sigma^2_{sp} \to 0$, INGARCH(1,1)/Short model
  	\end{itemize}
  \end{frame}
  
  \begin{frame}
  	\frametitle{SPINGARCH Model as Stochastic Difference Equation}

  	\begin{equation}
  		\frac{\lambda(s_i,t)-\lambda(s_i,t-1)}{\Delta t}=d-\chi \lambda(s_i,t-1)+\eta Z(s_i,t-1)
  	\end{equation}
  	\begin{itemize}
	\item 	Change in violence due to exogeneous $d$, natural decay, $\chi$, and excitement, $\eta$
	\item Assume each time period, exogeneous impact is stochastic and spatially correlated yields SPINGARCH

  	
  		\begin{equation}
  		\frac{\lambda(s_i,t)-\lambda(s_i,t-1)}{\Delta t}=\exp(Y(s_i,t))-\chi \lambda(s_i,t-1)+\eta Z(s_i,t-1)
  		\end{equation}
  		\item Change in intensity due to three factors, CAR, natural decay, and excitement
  		\end{itemize}
  	
  \end{frame}
  
    
  
  \begin{frame}
  	\frametitle{Data Realizations}
  	50 Spatial Observations on $\mathbb{R}^1$, 100 Temporal Observations
\begin{figure}[t!]
	\centering
	\includegraphics[width=0.475\textwidth]{SPINGARCHPlot}
	\hfill
	\includegraphics[width=0.475\textwidth]{SingePlot}
\end{figure}
 \begin{align}
 & \lambda(s_i,t) =\exp \left[ Y(s_i,t) \right] + 0.1 \mbox{ Z}(s_i,t-1)+ \mbox{0.4 } \lambda(s_i,t-1)\nonumber\\
 & Y(s_i,t)|\boldsymbol{Y}(N_i)  \sim \mbox{Gau}(\mu(s_i,t),0.5)\nonumber\\ 
 & \mu(s_i,t)  = 0+ 0.49 \sum_{s_j \in N_i} \{Y(s_j,t)\}\label{eq:SEPoissonGen}.
 \end{align}
  \end{frame}
  
  
  
  \begin{frame}
  	\frametitle{SPINGARCH Model - Parameter Space}
  	\begin{itemize}
  		\item $Z(s_i,t)|Y(s_i,t),\mathcal{H}_{Z(s_i)}\sim\mbox{Pois }(\lambda(s_i,t))$ where $\mathcal{H}_{Z(s_i)}$ is history of process at location $s_i$
  		\begin{align}
  			\lambda(s_i,t)&=\exp \left[ Y(s_i,t) \right] + \eta Z(s_i,t-1) + \kappa \lambda(s_i,t-1)\nonumber\\
  			& Y(s_i,t)|\boldsymbol{Y}(N_i)\sim N(\mu(s_i,t),\sigma_{sp}^2) \nonumber\\
  			& \mu(s_i,t) = \alpha(s_i)+ \zeta \sum_{s_j \in N_i} \{Y(s_j,t)-\alpha(s_j)\} \nonumber.
  		\end{align}
  		\item $\zeta \in  (\psi_{(1)}^{-1},\psi_{(n)}^{-1})$ where $\psi_{(i)}$ is the $i$th largest eigenvalue of adjacency matrix
  		\item For stationarity, $\eta>0$, $\kappa>0$, $\eta+\kappa <1$
  		
  	\end{itemize}
  \end{frame}
  
    \begin{frame}
    	\frametitle{SPINGARCH Model as Markov Chain}
    	\begin{itemize}
    	\item Let $\boldsymbol{\lambda_t}=\left(\lambda(s_1,t),\lambda(s_2,t),\cdots,\lambda(s_{n_d},t)\right)^T$
    	\item $\boldsymbol{C}$ is $n_d \times n_d$ with $C(i,j)=\zeta$ if $s_j \in N_i$
    	\item $\boldsymbol{M}=\mbox{diag } \sigma^2_{sp}$
  \begin{align}
  & Z(s_i,t)|\lambda(s_i,t) \sim \mbox{Pois}(\lambda(s_i,t)) \label{eq:timeseries2} \nonumber\\
  & E[Z(s_i,t)]=\lambda(s_i,t)\nonumber\\
  & \boldsymbol{\lambda_t} = \exp(\boldsymbol{Y_t})+\eta \boldsymbol{Z_{t-1}}+\kappa \boldsymbol{\lambda_{t-1}}\nonumber\\
  & \boldsymbol{Y_t} \sim \mbox{Gau} (\boldsymbol{\alpha_t},(I_{{n_d},{n_d}}-\boldsymbol{C})^{-1}\boldsymbol{M})
  \end{align}
  \item Markov chain for $\boldsymbol{\lambda_t}$ on State space, $(\mathbb{R}^{+})^{n_d}$
  \end{itemize}
    \end{frame}
    
    \begin{frame}
    	\frametitle{Impact of Initial Condtions and Recursion}
    	By recursion
    	\begin{align*}
    	& [\lambda(s_i,t)|\lambda(s_i,0)=B] = \exp(Y(s_i,t))+\kappa \lambda(s_i,t-1) + \eta Z(s_i,t-1)\\
    	 =& \exp(Y(s_i,t))+\kappa \left[\exp(Y(s_i,t-1))+\kappa \lambda(s_i,t-2)\right.\\
    	 & \left.+ \eta Z(s_i,t-2)\right] + \eta Z(s_i,t-1)\\
    	&\cdots\\
    	 =&\sum_{k=0}^{t-1} \kappa^k\exp(Y(s_i,t-k)) +\sum_{k=0}^{t-1} \kappa^k\eta Z(s_i,t-k-1)+\kappa^t B. \label{eq:Recursion}
    	\end{align*}
    \end{frame}

   
   \section{SPINGARCH Stationarity and Model Properties}
   

   
    \begin{frame}
    	\frametitle{Geometric Ergodicity with Finite Moments}
    		Under the parameter space restriction, $\eta,\kappa\geq0$ and $\eta+\kappa<1$, the SPINGARCH (1,1) is geometrically ergodic and admits a unique stationary distributions that has finite first two moments.
\begin{proof}[Sketch of Proof]
\textbf{Meyn and Tweedie (15.0.1)} need to show aperiodic, $\phi$-irreducible and $\exists$ test function $V(.)$ such that  $E[V(\boldsymbol{\lambda}_{t+1})|\boldsymbol{\lambda}_t=\boldsymbol{\lambda_*}]\leq \psi V(\boldsymbol{\lambda_*})+L \mbox{ I}(\boldsymbol{\lambda_*} \in C)$ holds where $\psi \in (0,1)$, $L \in (0,\infty)$ and $I(.)$ is the indicator function and $C$ is a petite set.
\newline

\textbf{Basic Idea}:  With positive probability, $\exists$ a realization $Z(s_i,1)=Z(s_i,2)=\cdots=Z(s_i,t-1) =0$.  Along that chain, $P(\lambda(s_i,t))\in A=P(\exp(Y(s_i,t))\in A-\kappa^t B)$.  If $\kappa^T B > \sup A$ run chain longer.
	\end{proof}
    \end{frame}
      \begin{frame}
      	\frametitle{Geometric Ergodicity with Finite Moments}
      	Under the parameter space restriction, $\eta,\kappa\geq0$ and $\eta+\kappa<1$, the SPINGARCH (1,1) is geometrically ergodic and admits a unique stationary distributions that has finite first two moments.
      	\begin{proof}[Sketch of Proof Cont.]
      	Test function $V(\lambda)=1+\lambda^2$ works for $E[V(\boldsymbol{\lambda}_{t+1})|\boldsymbol{\lambda}_t=\boldsymbol{\lambda_*}]\leq \psi V(\boldsymbol{\lambda_*})+L \mbox{ I}(\boldsymbol{\lambda_*} \in C)$.
      	\newline
      	
      	$\implies$ Unique stationary distribution, goes to geometrically fast.  Specific choice of $V(.)$ gives (at least) finite first two moments (can be extended likely as in Fokianos, 2009.)
      	
      	\end{proof}
      \end{frame}
      

      

     
       \begin{frame}
       	\frametitle{Increased Modeling Flexibility with SPINGARCH(1,1)}
       \begin{empheq}[box=\othermathbox]{align}
       	& Z(s_i,t)  \sim \mbox{Pois }(\lambda(s_i,t))\nonumber\\
       	& \lambda(s_i,t)=\exp \left[ Y(s_i,t) \right] + \eta Z(s_i,t-1) + \kappa E\left[Z(s_i,t-1)\right]\nonumber\\
       	& Y(s_i,t)|\boldsymbol{Y}(N_i) \sim N(\mu(s_i,t),\sigma_{sp}^2) \nonumber\\
       	& \mu(s_i,t) = \alpha(s_i)+ \zeta \sum_{s_j \in N_i} \{Y(s_j,t)-\alpha(s_j)\} \nonumber
       \end{empheq}
       
       	Define $\Sigma_{i,j}$ as $i,j$ entry of $(I_{{n_d},{n_d}}-\boldsymbol{C})^{-1}\boldsymbol{M}$
     \begin{align}
     & E\left[Z(s_i,t)\right]=\frac{1}{1-\eta-\kappa}\exp(\alpha+\frac{\Sigma_{1,1}}{2})\\
     &\small \mbox{Var }(Z(s_i,t))=\frac{1}{1-(\kappa+\eta)^2} \mbox{Var }(\exp(Y(s_i,t)))+\frac{1-\kappa^2-2\kappa \eta}{1-(\kappa+\eta)^2}E(Z(s_i,t))
    \end{align}
       \end{frame}
     
       \begin{frame}
       	\frametitle{Increased Modeling Flexibility with SPINGARCH(1,1)}
       	Temporal Covariance: 
       	\begin{equation}
       	\mbox{Cov }(Z(s_i,t),Z(s_i,t-1)= (\eta+\kappa)\mbox{Var}(Z(s_i,t))-\kappa E\left[Z(s_i,t)\right]
       	\end{equation}
       	\pause
       	\textbf{Recall}: Var-Mean Ratio at 2 $\implies$ Lag-1 correlation is between 0.5 and $\sqrt{1/2}$ for INGARCH(1,1)
       	\pause
       	\newline
       	\newline
       	Let $\kappa=0$ (SPINGARCH(0,1)), Var-Mean Ratio at 2
       	\begin{align}
       	& \implies 2= \frac{\mbox{Var} \left(\exp(Y(s_i,t))
       		\right)}{(1-\eta)^2 E\left[Z(s_i,t)\right]} + \frac{1}{1-\eta^2}\\
       	& \mbox{Cor }(Z(s_i,t),Z(s_i,t-1)=\eta
       	\end{align}
       	$\forall \eta \in (0,\sqrt{1/2})\quad \exists  \alpha,\sigma_{sp}^2$ such that equality holds
       	
       	
       \end{frame}
       
       \begin{frame}
       	\frametitle{Spatial Correlation}
       	\begin{align}
\small\mbox{Corr}(Z(s_i,t),Z(s_j,t))  = \frac{\left(\exp(\Sigma_{i,i}+\Sigma_{i,j}) -\exp(\Sigma_{i,i})\right)}{ \exp(2\Sigma_{i,i})-\exp(\Sigma_{i,i})+\exp(-\alpha+\frac{\Sigma_{i,i}}{2})\frac{1}{1-(\kappa+\eta)}}\label{eq:SpatCorr}\nonumber
       	\end{align}
\begin{figure}[t!]
	\centering
	\includegraphics[width=0.4\textwidth]{SpatialCorr}
\end{figure}
\centering
$\eta=.3$, $\sigma^2_{sp}=.5$, $4 \times 4$ to $15 \times 15 $ size lattice
       \end{frame}
       

 \section{Inference}
   \begin{frame}
   	\frametitle{Inference}
   	\begin{itemize}
   		\item Likelihood roots for INGARCH(1,1) easily found, asymptotically Gaussian
   		\item Inclusion of latent process in SPINGARCH(1,1) complicates
		\item $\boldsymbol{\theta}\equiv (\eta,\alpha,\zeta,\sigma^2_{sp})$

   	 \begin{align}
   	 \pi(\boldsymbol{\theta} | \boldsymbol{Z},\boldsymbol{Y})\propto &  \prod_t\pi(\boldsymbol{Z}_t|\boldsymbol{\lambda}_t)\pi(\boldsymbol{\lambda}_t|\boldsymbol{\lambda}_{t-1},\boldsymbol{Z}_{t-1},\boldsymbol{\theta} ,\boldsymbol{Y}_t)\pi(\boldsymbol{Y}_t|\boldsymbol{\theta} )
   	 \pi(\boldsymbol{\theta})
   	 \end{align}
   	 
   	 \begin{align}
   	 \pi(\boldsymbol{Y} | \boldsymbol{Z},\boldsymbol{\theta})\propto &  \prod_t\pi(\boldsymbol{Z}_t|\boldsymbol{\lambda}_t)\pi(\boldsymbol{\lambda}_t|\boldsymbol{\lambda}_{t-1},\boldsymbol{Z}_{t-1},\boldsymbol{\theta} ,\boldsymbol{Y}_t)\pi(\boldsymbol{Y}_t|\boldsymbol{\theta} ).
   	 \end{align}
   	 	\end{itemize}
   \end{frame}
 
 \begin{frame}
 	\frametitle{Efficient Bayesian Inference}
 	 \begin{align}
 	 \log(\boldsymbol{Y}|\boldsymbol{\alpha},\sigma_{sp},\zeta) &
 	 \propto \frac{1}{2} \log | \Sigma_f^{-1}(\theta)|\nonumber\\
 	 & - \frac{1}{2}(Y-\alpha)^T\Sigma_f^{-1}(\theta)(Y-\alpha) \label{eq:log Y},
 	 \end{align}
 	\begin{itemize}
 		\item  $\Sigma_f^{-1}\equiv\left(I_{n_d \times T,n_d \times T}-I_{t,t}\otimes \boldsymbol{C}\right)^{-1}I_{t,t}\otimes \boldsymbol{M}$
 		\item $\log|\Sigma^{-1}(\theta)|=\frac{n_d}{2 \log\sigma_{sp}^2}+\log|I_{n_d,n_d}-\zeta N|$
 		\item  Letting $V \Lambda V^T$ be the spectral decomposition of $N$ we have $|I_{n_d,n_d}-\zeta N|=|V| |I_{n_d,n_d}-\zeta \Lambda| |V^T|=\prod_{j=1}^{n_d}\left(1-\zeta \chi_j\right)$ where $\chi_j$ are the eigenvalues of the neighborhood matrix
 		\begin{align}
 		\log | \Sigma_f^{-1}(\theta)|&  = T \log | \Sigma^{-1}(\theta)|\\
 		& \propto \frac{n_d \times T}{\log\sigma_{sp}^2}+ T \sum_{j=1}^{n_d}(1-\zeta\chi_j) \label{eq:eig}
 		\end{align}
 	\end{itemize}
 \end{frame}
 

  \section{Simulation}
\begin{frame}
	\frametitle{Impacts of Misspecificaiton}
	\begin{align}
	& Z(s_i,t)  \sim\mbox{Pois}(\lambda(s_i,t))\nonumber\\
	& \lambda(s_i,t) =\exp \left[ Y(s_i,t) \right] + 0.66 Z(s_i,t-1)\nonumber\\
	& Y(s_i,t)|\boldsymbol{Y}(N_i)  \sim \mbox{Gau}(\mu(s_i,t),0.5)\nonumber\\ 
	& \mu(s_i,t)  = 0+ 0.49 \sum_{s_j \in N_i} \{Y(s_j,t)\}\label{eq:SEPoissonGen}.
	\end{align}
	\begin{figure}[!htp]
		\centering
		\includegraphics[width=0.5\linewidth, height=0.5\textheight]{SECARSim2}
	\end{figure}
\end{frame}


\begin{frame}
\frametitle{Generating Mechanism is SPINGARCH(0,1)}

	\begin{itemize}
		\item Fit to SPINGARCH(0,1), SPINGARCH(1,0), and INGARCH(1,1)
		\item Vague, proper priors, e.g $\eta\sim \mbox{Unif}(0,1)$, $\zeta \sim \mbox{Unif}(0,.5)$, $\sigma_{sp}\sim(Cau)^+(0,1)$, $\alpha \sim \mbox{Gau}(0,100)$
			\item Model assessment using posterior predictive P values 
			\begin{itemize}
			\item Pick ancillary statistic, $T(.)$ and calculate $T(\boldsymbol{Z})$
				\item for  $m=1...M$, draw a value of $\boldsymbol{\theta_m}$ according to $\pi(\boldsymbol{\theta}|\boldsymbol{Z})$
				\item Simulate $Z^*(\boldsymbol{s_i,t})_m$ of the same dimension as $\boldsymbol{Z}$ and compute $T(\boldsymbol{Z}^*_m)$
				\item Compute $\frac{1}{M}\sum_{m=1}^M I\left[T(\boldsymbol{Z}^*_m) > T(\boldsymbol{Z}) \right]$
				\end{itemize}
	\end{itemize}
\end{frame}




\begin{frame}
	\frametitle{Simulation and Estimation}
	\begin{itemize}
		\item SPINGARCH(0,1) 95\% credible intervals: $\alpha \in (-0.24,0.1)$, $\sigma^2 \in (0.46,0.59)$, $\zeta \in (0.486,0.492)$, and $\eta \in (0.64,0.67)$
		\item  SPINGARCH(1,0) 95\% credible intervals: $\alpha \in (-0.54,-0.2)$, $\sigma^2 \in (0.96,1.2)$, $\zeta \in (0.47,0.48)$, and $\kappa \in (0.65,0.67)$
	\begin{table}
		\begin{center}
				\begin{tabular}{ |c|c|c| } 
					\hline
					& SPINGARCH(1,0) & SPINGARCH(0,1)\\
					\hline 
					$p_1$ - Moran's I  & .05 & .46 \\
					$p_2$ - Var to Mean   & .99 & .65 \\
					$p_3$ - Lag 1 Corr  & .45 & .60\\ 
					\hline
				\end{tabular}
			\end{center}
		\end{table}
	\end{itemize}
\end{frame}

  \section{Burglaries in South Side of Chicago}
 \begin{frame}
 	\frametitle{Burglaries in South Side of Chicago}
   \begin{figure}[ht]
   	\begin{minipage}[b]{0.47\linewidth}
   		\centering
   		\includegraphics[width=\textwidth]{AllChi2}
   		\caption{Aggregated Burglaries}
   		\label{fig:a}
   	\end{minipage}
   	\hspace{0.5cm}
   	\begin{minipage}[b]{0.47\linewidth}
   		\centering
   		\includegraphics[width=\textwidth]{RacialSeg}
   		\caption{Racial Segregation}
   		\label{fig:b}
   	\end{minipage}
   	\end{figure}
 \end{frame}
 
\begin{frame}
	\frametitle{Burglaries South Side of Chicago}
	\begin{figure}[ht]
		\centering
		\includegraphics[width=7cm]{ChiBurg}
		\label{fig:SCSEMExample}
	\end{figure}
	\begin{itemize}
		\item Crime data from city of Chicago
		\item 72 months (2010-2015), 552 locations (Census block groups)
		\item Demographic data from Census bureau
	\end{itemize}
\end{frame}		

   \begin{frame}
   	\frametitle{SPINGARCH(1,1) Model}
   	
   	\begin{align}
   	& Z(s_i,t) \sim \mbox{Pois}(\lambda(s_i,t)) \label{eq:SPINGARCHCHI} \\
   	& \boldsymbol{\lambda_t} = \exp(\boldsymbol{Y_t}+\boldsymbol{U})+\eta \boldsymbol{Z_{t-1}}+\kappa \boldsymbol{\lambda_{t-1}}\nonumber\\
   	& \boldsymbol{Y_t} \sim \mbox{Gau}\textit{} (\boldsymbol{0},\sigma_{ind}^2 \boldsymbol{I}_{{n_d},{n_d}})\nonumber\\
   	& \boldsymbol{U} \sim \mbox{Gau} (\boldsymbol{\alpha},\sigma^2_{sp}[(\boldsymbol{N}-\boldsymbol{C})]^{-1})\nonumber
   	\end{align}
   	\begin{itemize}
  
   		\item WCAR specification (Weighted conditional variance of CAR)
   		\item $\boldsymbol{N}$ is $\mbox{diag}(|N_{i}|)$, $\implies \zeta \in(-1,1)$, fix at $\zeta=0.999$
 
   	\end{itemize}
   	\begin{align}
   	\alpha_{s_i}=&\exp\left(\beta_0+\beta_{pop} \log(\mbox{Pop}_{s_i})\nonumber\right.\\
   	&\left.+\beta_{ym}\mbox{Young Men}_{s_i}+\beta_{wealth}\mbox{Wealth}_{s_i}+\beta_{unemp}\mbox{Unemp}_{s_i}\right)
   	\end{align}
   \end{frame}
   
    \begin{frame}
    	\frametitle{Impacts of Including Spatial Correlation}
    	
    \begin{table}[h]
    	\begin{center}
    		\begin{tabular}{ |c|c|c| } 
    			\hline
    			Parameter & SPINGARCH(1,1) & INGARCH(1,1) \\
    			\hline 
    			$\beta_0$ & (-3.3,-1.0)& (-4.2,-3.4) \\
    			$\beta_{pop}$ & (0.11,0.34) & (0.33,0.46)\\
    			$\beta_{ym}$ & \highlight{(-0.75,0.17)}& \highlight{(0.06,0.09)}\\
    			$\beta_{wealth}$&\highlight{(0.05,0.16)} & \highlight{(-0.04,0.01)}\\
    			$\beta_{unemp}$ & (0.006,0.07)& (0.002,0.03) \\
    			$\eta$ & \highlight{(0.04,0.07)} & \highlight{(0.22,0.24)}\\
    			$\kappa$ & (0.31,0.39)& (0.44,0.48)\\
    			$\sigma_{sp}^2$ & (0.40,0.54) & - \\
    			$\sigma_{ind}^2$& (0.40,0.47)& - \\
    			\hline
    		\end{tabular}
    	\end{center}
    \end{table}
    \end{frame}
    
 
 
 
  \begin{frame}
  	\frametitle{Model Assessment - Posterior Predictive Checks}
  	\begin{table}[h]
 \label{Table:Pvals} 
  		\begin{center}
  			\begin{tabular}{ |l|c|c| } 
  				\hline
  				& SPINGARCH(1,1) & INGARCH(1,1)\\
  				\hline 
  				$p_1$ - Moran's I Statistic& 0.43 & 0 \\
  				$p_2$ - Variance to Mean Ratio & 0.62 & 0\\
  				$p_3$ - Lag 1 Auto Correlation & 0.67 & 0.83 \\ 
  				\hline
  			\end{tabular}
  		\end{center}
  	\end{table}
  	\begin{itemize}
  		\item SPINGARCH(1,1) - observed maximum (p=.67), number of zeros (p=.49)
 
  	\end{itemize}
  \end{frame}
  
  
  \begin{frame}
  	\frametitle{Summary}
\begin{itemize}
	\item INGARCH(1,1) process unable to replicate second order properties of burglaries in Chicago, SPINGARCH(1,1) much more so
	\item Exogeneous covariates offer some structure for crime, but rarely, if ever, adequately account for all
	\item Failure to account for small scale spatial structure leads to differing conclusions - possible policy implications
	
\end{itemize}
  \end{frame}
 
 \begin{frame}
 	\frametitle{Future Work}	
 	\begin{itemize}
 	\item Impacts of aggregation
 	\item Laplace approximations greatly speed up SPINGARCH(0,1) - Can extend to SPINGARCH(1,1)?
 	\item Reaction Diffusion Self-Exciting Model from (Clark \& Dixon, 2018) does not fit in framework (temporally correlated errors)
 	\begin{itemize}
 		\item RDSEM captures reaction diffusion process of Short in Latent Process
 	\end{itemize}
 	\item Dropping self-excitement leads to SPDE with exact solution - Sparse approximation?
 	\end{itemize}
 \end{frame}
 
 \begin{frame}
 	\frametitle{Conclusion}
 	\begin{itemize}
 	\item SPINGARCH model has potential to model phenomena where there is expected data correlation and limited spatial correlation
 	\item Data model correlation manifests differently then latent correlation structure and should be accounted for accordingly
 	\item Although derived from crime and violence, potential use for suicides, weather, etc
 	\end{itemize}
 	
 	\Large Thank you for your time!
 	
 \end{frame}
 
 
 \appendix
 
 \begin{frame}
 	\frametitle{Post Talk Slides}
 	\begin{itemize}
 		\item Research Chapter 1 - SCSEM and RDSEM Models applied to Iraq Data, Laplace Approximation based exploration of posterior density of parameters
 		\item Research Chapter 2 - Extend SCSEM and put in context of other statistical models, apply to Burglaries in Chicago
 		\item Research Chapter 3 - Discovered bias in Laplace approximation for subset of parameter space.  Explains why bias occurs and how to fix (in some instances)
 	\end{itemize}
 \end{frame}
 
 \begin{frame}
 	\frametitle{Ch. 3 - An Extended Laplace Approximation Technique for Bayesian Inference}
 	
 		
 		\textit{Far better an approximate answer to the right question, which is often vague, than an exact answer to the wrong question, which can always be made precise} \textbf{The Future of Data Analysis.} Annals of Mathematical Statistics 33: 1-67 John W. Tukey 1962 
 \end{frame}
 
 \section{Difficulties with Off the Shelf Inference}
 
   \begin{frame}
   	\frametitle{SPINGARCH (0,1) - Spatially Correlated Self-Exciting Model}
   Previously used in inference on violence in Iraq
   
   	\begin{align}
   	& Z(s_i,t) \sim \mbox{Pois}(\lambda(s_i,t)) \label{eq:timeseries2} \\
   	& E[Z(s_i,t)]=\lambda(s_i,t)\\
   	& \boldsymbol{\lambda_t} = \exp(\boldsymbol{Y_t})+\eta \boldsymbol{Z_{t-1}}\\
   	& \boldsymbol{Y_t} \sim \mbox{Gau}\textit{} (\boldsymbol{\alpha_t},(I_{{n_d},{n_d}}-\boldsymbol{C})^{-1}\sigma^2_{sp}).
   	\end{align}
   \end{frame}
 

   \begin{frame}
   	\frametitle{Integration Free Technique}
   	Laplace Approximation to marginals
	\begin{align}
	\tilde{\pi}(\eta,\zeta,\sigma_{sp}^2,\alpha|\boldsymbol{Z})\propto \frac{\pi(\boldsymbol{Z}|\eta,\boldsymbol{Y})\pi(\boldsymbol{Y}|\alpha,\zeta,\sigma_{sp}^2)\pi(\zeta)\pi(\alpha)\pi(\sigma_{sp}^2)\pi(\zeta)}{\pi_G(\boldsymbol{Y}|\alpha,\eta,\zeta,\sigma_{sp}^2,\boldsymbol{Z})} \label{eq:INLA},
	\end{align}
	$\pi_G(\boldsymbol{Y}|\alpha,\eta,\zeta,\sigma_{sp}^2,\boldsymbol{Z})$ is Gaussian approximation from Taylor series expansion of full conditional
		\begin{align}
		\pi_G(\boldsymbol{Y}|\eta,\zeta,\sigma_{sp}^2,\boldsymbol{Z}) & \propto (2 \pi)^{n/2} \det(\Sigma(\theta))^{1/2} \exp\left(-\frac{1}{2}(\boldsymbol{Y})^t \Sigma^{-1}(\theta)\boldsymbol{Y} \nonumber 
		\label{eq:gausapprox}\right.\\&\left.+\sum_{s_i,t} f(\mu(s_i,t))(Y(s_i,t))+ 1/2 k (\mu(s_i,t))(Y(s_i,t))^2\right)
		\end{align} 
   \end{frame}
   
   
     \begin{frame}
     	\frametitle{Issues}
     	Evaluate $\pi_G(\boldsymbol{Y}|\eta,\zeta,\sigma_{sp}^2,\boldsymbol{Z})$ at mode $\boldsymbol{Y}=\boldsymbol{\mu^*}$:
     	\begin{align}
     	\log(\pi(\boldsymbol{\theta}|\boldsymbol{Z}))&\propto \log\pi(\boldsymbol{Z}|\eta,\boldsymbol{\mu}^*) +\log\pi(\boldsymbol{\boldsymbol{\mu}^*}|\zeta,\sigma^2_{sp},\alpha)-  1/2\log|\Sigma(\boldsymbol{\theta})| \nonumber\\& + \log \pi(\boldsymbol{\theta}) + 1/2\log |\Sigma(\boldsymbol{\theta})+\mbox{diag }k(\mu(s_i,t)^*)|
     	\end{align}
     	\begin{itemize}
     		\item Even for small $T$, $\pi(\theta|\boldsymbol{Z})$ appears to be Gaussian
     		\item $\pi_G(\boldsymbol{Y}|\alpha,\eta,\zeta,\sigma_{sp}^2,\boldsymbol{Z})$, not a good approximation
     		\item $\sigma_{sp}^2$ in particular appears biased (low)
     		\item Bias increases as $\eta$ or $\sigma^2_{sp}$ increases
     	\end{itemize}
     \end{frame}
    \begin{frame}
    	\frametitle{Likelihood}
    	Full likelihood has latent dimensionality that increases as $n$ increases:
    	\begin{align}
    	L(\eta,\alpha,\zeta,\sigma_{sp}^2|\boldsymbol{Z}) \propto & \int_{\boldsymbol{\Omega}_y} \prod_{i=1}^{n}\prod_{t=1}^{T} \exp(-\eta Z(s_i,t-1)-\exp(Y(s_i,t)))\nonumber\\
    	&\times \left(\eta Z(s_i,t-1)+\exp(Y(s_i,t))\right)^{Z(s_i,t)} d\mu_{\boldsymbol{Y}}\label{eq:FullLikelihood}.
    	\end{align}
    	Given data, $\boldsymbol{Z}$:
    	\begin{align}
    	L(\eta,\alpha,\zeta,\sigma_{sp}^2|\boldsymbol{Z}) &\propto  \prod_{t=1}^{T}\int_{\boldsymbol{\Omega}_{y_t}} \prod_{i=1}^{n} \exp(-\eta Z(s_i,t-1)-\exp(Y(s_i,t)))\nonumber\\
    	&\times \left(\eta Z(s_i,t-1)+\exp(Y(s_i,t))\right)^{Z(s_i,t)} d\mu_{\boldsymbol{Y_t}}\label{eq:FullLikelihood2}
    	\end{align}
    	Still high dimensional, intractable
    \end{frame}
    
    
       \begin{frame}
       	\frametitle{Standard Laplace Approximation}
		\begin{align}
		&\int_{\boldsymbol{\Omega}_{y_t}} \prod_{i=1}^{n} \exp(-\eta Z(s_i,t-1)-\exp(Y(s_i,t)))\nonumber\\
		&\times \left(\eta Z(s_i,t-1)+\exp(Y(s_i,t))\right)^{Z(s_i,t)} d\mu_{\boldsymbol{Y_t}}\nonumber
		\end{align}
		Let
		\begin{align} & g(\boldsymbol{Y_t})=\log\pi(\boldsymbol{Y_t}|\boldsymbol{\theta})+\log(\boldsymbol{Z_t}|\boldsymbol{\theta},\boldsymbol{Z_{t-1}})\\
		& \frac{\partial g}{\partial Y(s_i,t)}=g_i\\
		& \frac{\partial g}{\partial Y(s_i,t)\partial Y(s_j,t)}= g_{ij}\\
		& g_{\boldsymbol{Y}\boldsymbol{Y}}= \mbox{Hessian matrix of }g\\
		& g^{ij}= (i,j)\mbox{th element of inverse Hessian}\\
		& \hat{g}^{ij} \mbox{ evaluated at mode of }\pi(\boldsymbol{Y}|.)
		\end{align}
             \end{frame}
             
         	\begin{frame}
         		\frametitle{Standard Laplace}
				\begin{equation}
         		M_t=\int \exp(-g(\boldsymbol{Y}_t))dY_t\nonumber
         		\end{equation}
         		 \begin{itemize}
         			\item Multivariate Taylor Series expansion of $g$ about unique minimum followed by Taylor Series expansion of $\exp$ about zero yields
         			
         			\begin{align}
         			\boldsymbol{M}&=\exp(-\hat{g})\Big|\frac{\hat{g}_{\boldsymbol{Y}\boldsymbol{Y}}}{2\pi}\Big|E\left[1-\frac{1}{3!}\hat{g}_{i,j,k}U(s_1,t)U(s_2,t)U(s_3,t)\right.\nonumber\\
         			&\left.-\frac{1}{4!}\hat{g}_{i,j,k,l}U(s_1,t)U(s_2,t)U(s_3,t)U(s_4,t)-\cdots\right]
         			\end{align}
         			\item $\boldsymbol{U}\sim\mbox{Gau}(\boldsymbol{0},\hat{g}_{\boldsymbol{Y}\boldsymbol{Y}})$
         			\item $\hat{g}_{i,j,k,l}=0$ unless $i=j=k=l$, $E[U(s_1,t)^4]=3({g}^{ii})^2$
         		\end{itemize}
         	\end{frame}
         	
         	\begin{frame}
         		\frametitle{First Three Truncated Terms}
         		\begin{align}
         		& \frac{1}{4!}\hat{g}_{i,j,k,l}E[U(s_1,t)U(s_2,t)U(s_3,t)U(s_4,t)]=\frac{1}{8} \sum_{i}\hat{g}_{iiii}(\hat{g}^{ii})^2\\
         		& =  -\frac{1}{72}\sum_{i,j\leq i}\hat{g}_{iii}\hat{g}_{jjj}\left(6 \left(\hat{g}^{ij}\right)^3+9 \hat{g}^{ii}\hat{g}^{jj}\hat{g}^{ij}\right)\\
         		& = \frac{1}{48}\sum_{i}\hat{g}_{iiiiii}(\hat{g}^{ii})^4
         		\end{align}
         	Issues when $\hat{g}^{ii}>1$  
         	
         	Terms increase as $\eta$ and $\sigma_{sp}^2$ increase
         	\end{frame}

    
  \begin{frame}
    	\frametitle{Extended Laplace Approximation}
    	(Shun \& McCullagh, 1995) and (Evangelou et al. 2011)
    			\begin{equation}
    			M_t=\int \exp(-g(\boldsymbol{Y}_t))dY_t\nonumber
    			\end{equation}
    			\begin{itemize}
		\item Taylor series expansion of $g(.)$, $\boldsymbol{M_t}=\exp(-\hat{g})E\exp(-\hat{g}_i Y(s_i)-\hat{g}_{ij}Y(s_i)Y(s_j)/2!-\hat{g}_{ijk}Y(s_i)Y(s_j)Y(s_k)/3!-\cdots)$
		
		\item  $\log M_t$ is joint cumulant-generating function of $Y(s_i),Y(s_i)Y(s_j),Y(s_i)Y(s_j)Y(s_k),$ etc.
		
		\item Using Shun \& McCullagh (1995) for $\log M$ 

    	\begin{align}
    	\log M & \propto -\hat{g}-\frac{1}{2}|\hat{g}_{\boldsymbol{YY}}|-\sum_{t}\sum_{i}\frac{1}{8}\hat{g}_{iiii}(\hat{g}^{ii})^2- \nonumber\\
    	&\sum_{t}\sum_{i}\frac{1}{48}\hat{g}_{iiiiii}(\hat{g}^{ii})^4 +
    	\frac{1}{72}\sum_{t}\sum_{i,j\leq i}\hat{g}_{iii}\hat{g}_{jjj}\left(6 \left(\hat{g}^{ij}\right)^3+9 \hat{g}^{ii}\hat{g}^{jj}\hat{g}^{ij}\right) \nonumber
    	\end{align}
		\end{itemize}

  \end{frame}
 
     
 
 
 \begin{frame}
 	\begin{align}
 	&\tilde{\pi}(\eta,\zeta,\sigma_{sp}^2,\alpha|\boldsymbol{Z})\propto \frac{\pi(\boldsymbol{Z}|\eta,\boldsymbol{Y})\pi(\boldsymbol{Y}|\alpha,\zeta,\sigma_{sp}^2)\pi(\zeta)\pi(\alpha)\pi(\sigma_{sp}^2)\pi(\zeta)}{\pi_E(\boldsymbol{Y}|\alpha,\eta,\zeta,\sigma_{sp}^2,\boldsymbol{Z})} \label{eq:INLA2},
 	\end{align}
 	\begin{algorithm}[H]
 		\begin{algorithmic}[1]
 			\STATE Fix $\boldsymbol{\theta}=\boldsymbol{\theta_m}$ (Near Moment Based Estimates of $\boldsymbol{\theta}$)
 			\STATE Repeatedly solve $\left(\Sigma^{-1}(\boldsymbol{\theta})+\mbox{diag  } g^{ii}(\boldsymbol{\mu}_n)\right){\mu}_{n+1}=f(\boldsymbol{\mu}_n)$, yielding $\boldsymbol{\mu^*}$ mode of $\pi_G(\boldsymbol{Y}|\boldsymbol{\theta})$
 			\STATE Evaluate $g_{iii},g_{iiii},g_{iiiiii},g^{ii},g^{ij}$ at $\boldsymbol{\mu^*}$ and $\log \pi(\boldsymbol{\theta})$ to approximate $\log M$
 			\STATE Numerically estimate Hessian of $\boldsymbol{\theta}$
 			\STATE Conduct Newton-Raphson update of $\boldsymbol{\theta}$
 			\STATE Repeat until convergence
 		\end{algorithmic}
 		\caption{pseudocode for the calculation of marginal density of $\boldsymbol{\theta}$}
 		\label{alg:seq}
 	\end{algorithm}
 \end{frame}
 
  
  \begin{frame}
  	\frametitle{When it Works}
  	10 $\times$ 10 Lattice, 100 Time Points, $\zeta=.245,\alpha=0$
  		\begin{table}[h]
  			\begin{center}
  				\begin{tabular}{ |l|c|c| } 
  					\hline
  					& $\eta=.1$, $\sigma_{sp}^2=.4$&$\eta=.4$, $\sigma_{sp}^2=.6$\\
  					\hline
  					Relative Bias in LA(1) & .09 & .25 \\ 
  					Time to Fit LA(1) (min.)& 2 & 3 \\
  					\hline
  					Extended LA Without 6th Order & .03 & .07 \\
  					Extended LA With 6th Order& .03 & .02 \\
  					Time to Fit Extended LA & 3 & 5 \\
  					\hline
  					MCMC & .02 & .01 \\
  					Time to Fit MCMC & 50-65 & 50-65 \\
  					\hline
  				\end{tabular}
  			\end{center}
  			\end{table}
  		 
  \end{frame}
  
  \begin{frame}
  	\frametitle{When it Doesn't}
  		\begin{center}
  			\begin{tabular}{ |c|c| } 
  				\hline
  				&  $\eta=.7$, $\sigma_{sp}^2=1$\\
  				\hline
  				Relative Bias in LA(1) & .32\\ 
  				Time to Fit LA(1) (min.)& 5\\
  				\hline
  				Extended LA Without 6th Order & .14\\
  				Extended LA With 6th Order& .25\\
  				Time to Fit Extended LA & 5\\
  				\hline
  				MCMC & .04\\
  				Time to Fit MCMC & 75-244\\
  				\hline
  			\end{tabular}
  		\end{center}
  			\begin{itemize}
  				\item Examine $\frac{1}{48}\sum_{i}\hat{g}_{iiiiii}(\hat{g}^{ii})^4$, $\frac{1}{8}\sum_{i}\hat{g}_{iiii}(\hat{g}^{ii})^2$
  				\item Higher order terms $(\hat{g}^{ii})^k$ for $k>2$
  				\item $\sigma_{sp}^2=1, \eta=.7$ yield $\hat{g}^{ii}>1$
  			\end{itemize}
  \end{frame}
  
  \begin{frame}
  	\frametitle{When it Works and When it Doesn't}	\begin{table}[h]
  		\begin{center}
  			\begin{tabular}{ |c|c|c|c| } 
  				\hline
  				95\% CI Cover Parameter&$0<\hat{g}^{ii}<.5$&$ .5<\hat{g}^{ii}<2$ & $2<\hat{g}^{ii}$\\
  				\hline
  				LA(1)& 12/14 & 0/39 & 0/37\\
  				Extended LA & 13/14 & 36/39 & 5/37\\
  				\hline
  			\end{tabular}
  		\end{center}
  
  	\end{table}
  	
  		\begin{figure}[!htp]
  			\centering
  			\includegraphics[width=0.7\linewidth, height=0.6\textheight]{Fitting3}

  		\end{figure}
  \end{frame}

\begin{frame}
\frametitle{When it Works and When it Doesn't}
$\hat{g}^{ii}$ terms for $\eta=.4$, $\sigma_{sp}^2=.6$ and $\eta=.7$, $\sigma_{sp}^2=1$
		\begin{figure}[ht]
			\begin{minipage}[b]{0.4\linewidth}
				\centering
				\includegraphics[width=\textwidth]{Works1}
				\caption{$\frac{1}{48}\sum_{i}\hat{g}_{iiiiii}(\hat{g}^{ii})^4=-41$}
				\label{fig:a}
			\end{minipage}
			\hspace{0.5cm}
			\begin{minipage}[b]{0.4\linewidth}
				\centering
				\includegraphics[width=\textwidth]{Doesnt}
				\caption{$\frac{1}{48}\sum_{i}\hat{g}_{iiiiii}(\hat{g}^{ii})^4=-440$}
				\label{fig:b}
			\end{minipage}
		\end{figure}
\end{frame}


  \begin{frame}
  	\frametitle{Violent Crime in Chicago Aggregated By Neighborhood}
  	
  	
  Weekly from (December 28 2014 - January 2, 2016)
  	
  	
  	\begin{figure}[!htp]
  		\centering
  		\includegraphics[width=0.5\linewidth, height=0.5\textheight]{CrimeMap}
  	\end{figure}
  	
  \end{frame}
  
   \begin{frame}
   	\frametitle{Model}
   	SPINGARCH(0,1) or Spatially Correlated Self-Exciting model
   		\begin{align}
   		& Z(s_i,t) \sim \mbox{Pois}(\lambda(s_i,t)) \label{eq:timeseries3} \\
   		& E[Z(s_i,t)]=\lambda(s_i,t)\\
   		& \boldsymbol{\lambda_t} = \exp(\boldsymbol{Y_t})+\eta \boldsymbol{Z_{t-1}}\\
   		& \boldsymbol{Y_t} \sim \mbox{Gau} (\boldsymbol{\alpha_t},(I_{{n_d},{n_d}}-C)^{-1}M)
   		\end{align}
   		\begin{itemize}
   		\item Assume both spatial and temporal covariates $\alpha(s_i,t)=\beta_0 + \beta_{temp} \mbox{ Temp}(t) + \beta_{pop} \mbox{ Pop}(s_i)$ 
   		\item Vague proper priors for all parameters
   		\end{itemize}
   \end{frame}
   
    \begin{frame}
    	\frametitle{Posterior Median}
    		\begin{table}[h]
    			\begin{center}
    				\begin{tabular}{ |c|c|c|c|c|c|c| } 
    					\hline
    					Point Estimates & $\sigma_{sp}^2$ & $\zeta$ & $\eta$ & $\beta_0$ & $\beta_1$ & $\beta_2$\\
    					\hline
    					LA & .38 & .180 &.50& -5.6 & .17 & .50 \\ 
    					Extended LA & .52 & .179 &.50& -5.6 & .18 & .49\\
    					MCMC & .50 & .179 & .50 & -5.6 & .18 & .49\\
    					\hline
    				\end{tabular}
    			\end{center}
    		\end{table}
    	\begin{itemize}
    	  \item Extended LA fit without 6th order term
    	  \item Stan - 3 chains, 15000 samples, no evidence of non-convergence, 3 hours run in parallel
    	  \item Extended LA/LA - 15 minutes 
    	  \end{itemize}
    \end{frame}

  \begin{frame}
  	\frametitle{$\hat{g}^{ii}$ Terms at Posterior Median}
  	
  	
  	\begin{figure}[!htp]
  		\centering
  		\includegraphics[width=0.5\linewidth, height=0.5\textheight]{ChiGii}
  	\end{figure}
  	
  \end{frame}



  	\begin{frame}
  		\frametitle{Comparison of 95\% Credible Intervals}
	  		\begin{table}
  				\begin{center}
  					\begin{tabular}{ |c|c|c|c| } 
  						\hline
  					& $\sigma_{sp}^2$ & $\zeta$ & $\eta$\\
  						\hline 
  						Extended LA& (.43,.61) & (.176,.182) &(.47,.53\\
  						MCMC & (.42,.59) & (.176,.182) & (.47,.53)\\
  						\hline
  					\end{tabular}
  				\end{center}
  			\end{table}
  				\begin{table}
  					\begin{center}
  						\begin{tabular}{ |c|c|c|c| } 
  							\hline
  							&  $\beta_0$ & $\beta_1$ & $\beta_2$\\
  							\hline 
  							Extended LA& (-6.3,-4.9)& (.09,.27) & (.42,.55)\\
  							MCMC &  (-6.3,-5.0) & (.09,.27) & (.42,.56)\\
  							\hline
  						\end{tabular}
  					\end{center}
  				\end{table}
  			
  	\end{frame}
  	
   \section{Summary}
  		\begin{frame}
  			\frametitle{Summary}
  			\begin{itemize}
  				\item Essential to examine magnitude of $\hat{g}^{ii}$ terms, if large, LA will have non-negligible issues
  				\item Extending Laplace approximation removes bias over wide range of parameter space and credible intervals comparable to MCMC
  				\item Pay price in front end coding and derivations
  			\end{itemize}
  		\end{frame} 

\begin{frame}
	\frametitle{Dissertation Contributions}
	\begin{itemize}
		\item Models for violence that are consistent with sociological theories on how violence spreads
		\item New class of models that extends existing INGARCH models to spatial-temporal problems and accurately capture beliefs on how violence and crime spreads
		\item Methodology for inference that is quick and relatively accurate 
	\end{itemize}
\end{frame}

\begin{frame}
	\frametitle{Remaining Gaps}
	\begin{itemize}
		\item Extended LA for SPINGARCH(1,1)
		\item Theory for RDSEM models (INGARCH with spatio-temporally correlated latent structure)
		\item Impacts of aggregation (SPDE Approach)
	\end{itemize}
\end{frame}

\begin{frame}
	\frametitle{Achievements}
	\begin{itemize}
		\item Chapter 1 - Best Paper Competition Winner, Accepted in AoAS
		\item Omar Bradley Fellowship
		\item Graduate College Emerging Leader
		\item Invited Speaker - NSA
		\item Contributed Talk - JSM
		\item Invited Revision for Hidden Markov Model behavior of Mountain Goats
		\item Current manuscript on multivariate HMM 
	\end{itemize}
\end{frame}

\end{document}






  
  \begin{frame}
  	\frametitle{Profile of $\sigma^2_{sp}$ for differing $\eta$ values}
  	\begin{figure}[!htp]
  		\centering
  		\includegraphics[width=0.5\linewidth, height=0.5\textheight]{SigPlots}
  	\end{figure}
  	Red is LA(1), Blue is Extended LA, $\eta\in \{0,.1,.3,.5,.7,.9\}$
  \end{frame}
  



